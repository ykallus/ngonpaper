\documentclass[11pt, oneside]{article}   	% use "amsart" instead of "article" for AMSLaTeX format
\usepackage{geometry}                		% See geometry.pdf to learn the layout options. There are lots.
\geometry{letterpaper}                   		% ... or a4paper or a5paper or ... 
%\geometry{landscape}                		% Activate for for rotated page geometry
%\usepackage[parfill]{parskip}    		% Activate to begin paragraphs with an empty line rather than an indent
\usepackage{graphicx}				% Use pdf, png, jpg, or eps� with pdflatex; use eps in DVI mode
								% TeX will automatically convert eps --> pdf in pdflatex		
\usepackage{amssymb}
\usepackage{amsmath}
\usepackage{amsthm}
\usepackage [autostyle, english = american]{csquotes}

\newtheorem{definition}{Definition}[section]
\newtheorem{theorem}{Theorem}[section]
\newtheorem{conjecture}{Conjecture}[section]
\newtheorem{lemma}{Lemma}[section]
\newtheorem{conj}{Conjecture}[section]
\newtheorem{corollary}{Corollary}[section]
\newtheorem{remark}{Remark}[section]
\newtheorem{proposition}{Proposition}[section]
\newtheorem{conditions}{Conditions}[section]

%\newcommand{\defv}[1]{\textbf{\textit{#1}}}



\title{The local optimality of the double lattice packing}
\author{Yoav Kallus and W\"oden Kusner}
%\date{}							% Activate to display a given date or no date

\begin{document}
\bibliographystyle{plain}
%\maketitle
%\abstract
%This paper shows that the dense double lattice construction of Kuperberg and Kuperberg is locally optimal for polygons in the full space of packings.


\section{The construction of the densest double lattice packing}

\subsection{Double lattices}

\begin{definition}
    A chord of a convex body $K$ is a line segment whose endpoints lie on the boundary of $K$.
    A chord is an affine diameter if there is no longer chord parallel to it.
\end{definition}

\begin{definition}
    An inscribed parallelogram is a half-length parallelogram in the direction $\theta$ if
    one pair of edges is parallel with the line through the origin at an angle $\theta$ above
    the $x$-axis and their length is half the length of an affine diameter parallel to them.
\end{definition}

Note that any two half-length parallelograms in the direction $\theta$ have equal
area, and we can define that area as a function $A(\theta)$ of the direction.

\begin{definition}
    A cocompact discrete subgroup of the Euclidean group consisting of translations
    and point reflections is a \text{double lattice} if it includes at least one
    point reflection.
\end{definition}

A double lattice is generated by a lattice and a point reflection, or alternatively
by three point reflections.

\begin{theorem}[Kuperberg and Kuperberg]\label{thmkk}%strict convexity is not necessary (see Mount)
    For a convex $K$, an admissible double lattice of smallest mean area has
    mean area $4\min_\theta A(\theta)$ and is generated by reflection about
    the vertices of a half-length parallelogram.
\end{theorem}

%By a sequence of approximations, this gives that for convex bodies $K$ that are not strictly convex, there
%exists a double lattice packing of maximal density that generated a minimum area extensive parallelogram in $K$.

Kuperberg and Kuperberg make use of extensive parallelograms, inscribed parallelograms with edge length greater than half the affine diameter in their edge directions, but make use of half-length parallelograms. Mount gives an explicit proof that, when minimizing the volume of an extensive parallelogram, it suffices to consider only the half-length parallelograms associated with affine diameters as a function of $\theta.$

From the definition of an extensive parallelogram, one observes that this is exactly the condition that produces an admissible, connected?tight? double lattice packing, and that the density of the packing increases as the area of the extensive parallelogram decreases.  Thus, there is a direct correspondence between the admissible motions of the double lattice packing and the motions of inscribed, extensive parallelograms. 

 For example, sliding motions of a double lattice packing that preserves density and contact correspond to motions of an inscribed extensive parallelogram that preserve area.

%include images.

For our purposes, it is illustrative to consider the algorithm used find the half-length parallelogram of minimal area not only as a method of finding the best double lattice packing, but also as a way to explore part of the configuration space of double lattices. More specifically, the behavior of the affine diameter is fairly well behaved but non-trivial as $\theta$ increases while the behavior of the vertices of the half parallelogram are described by an {\it interspersing property}; they are non-decreasing functions $[0,2\pi)\rightarrow [0,2\pi).$  %they are a non-decreasing function, they never back track....

We begin with a simple proposition:
\begin{proposition}
there is an affine diameter of a convex polygon $K$ in every direction that meets a vertex.
\end{proposition}

This is a consequence of the convexity $K$, for if a chord does not meet a vertex, it either lies within an edge or it meets two edges.  If it lies within an edge, it is not an affine diameter.  If it meets two edges, it is possible increase its length while parallel translating it in a non-decreasing direction of the cone defined by the two edges until it meets a vertex. 
%picture

From this, it is possible to determine an initial affine diameter in a particular direction.  Furthermore, this is done by geometrically satisfying procedure of extending a set of parallel rays in a from all vertices into the interior of $K$ and selecting the longest.  However, it does not give 

\subsection{Tracking the affine diameter }


Walking around the vertices


Aside from the degenerate sliding configurations, the affine diameter has a choice of a {\it moving end} and a {\it fixed end} with respect to an increase in $\theta.$

Parallel edges and non-uniqueness, a sliding motion.


\subsection{Tracking the affine diameter }

tracking the half length parallelogram  




\subsection{critical angles}
In order to generate the densest double lattice packing, one must also find the minimal area half length parallelogram.  This is done by moving between critical angles of the affine diameter where a vertex of the half-length parallelogram or both vertices of the affine diameter meet vertices of $K$.  





Tracking the motion via a parametrization of the slopes.

Note that these also correspond to the degenerate situations for our stability condition.


\subsection{stability condition}
%can be reworked...
At any particular configuration, as $theta$ varies, there is a relationship between the linear motions of the points given by the system of equations defining the half-length parallelogram.  That is, the 

Consider the vertex of the half length parallelogram.  As theta varies, 


We parametrize the motion of the vertices $\mathbf p_2$, $\mathbf p_3$,  $\mathbf p _5$, $\mathbf p_6$ relative to a motion of the moving end of the affine diameter $\mathbf p_4$.

The moving end of the affine diameter moves linearly, as do the vertices of the half-length parallelogram, as it they are constrained to be on the edges of $K$.  The length of the chords is also a linear function in $\theta$.   Therefore, their motion is described by their initial position plus some constant velocity motion ${\mathbf v_i}$, provided that the vertices of the half-length parallelogram and the moving end of the affine diameter remain on their initial edges.


\begin{equation}
    \begin{aligned}
&{\mathbf p}_i(t) = {\mathbf p}_i^{inital} + k_i t {\mathbf v_i}\\
&\text{satisfying} \\
&\frac{1}2 {\mathbf p}_4(t)-{\mathbf p}_1 = \mathbf p_3(t) - \mathbf p_2(t)\\
&\text{and}\\
&\frac{1}2 {\mathbf p}_4-{\mathbf p}_1 = \mathbf p_5(t) - \mathbf p_6(t)
    \end{aligned}
\end{equation}

This system can be solved for the rate constants $k_i$ which give conditions on the motion of the parallelogram.  We fix the affine diameter as a horizontal chord of length one, define variable motions of the point $\mathbf p_i$ at inclination $\phi_i$ and let the moving end of the affine diameter move at unit speed.  

Solving this system yields rate constants 
\begin{equation}
    \begin{aligned}
   & k_1 = 0\\
&k_2 = \frac{1}{2} \sin \left(\phi _3-\phi _4\right) \csc \left(\phi _1-\phi _3\right)\\
&k_3 = \frac{1}{2} \sin
   \left(\phi _1-\phi _4\right) \csc \left(\phi _1-\phi _3\right)\\
  &k_4=1\\
&k_5 =  \frac{1}{2} \sin \left(\phi _4-\phi _6\right) \csc
   \left(\phi _5-\phi _6\right)\\
&k_6 = \frac{1}{2} \sin \left(\phi _4-\phi _5\right) \csc \left(\phi _5-\phi
   _6\right)
    \end{aligned}
\end{equation}
%\begin{remark}
%Furthermore,  IMVT should say that ever point is met by an affine diameter as well, so every vertex is met by one.? Combining these satisfies the intuition that one may 
%\end{remark}



%continuous sweeping vs finding the minimum area intermediate configuration...
%can introduce the notion of stability here

%Having found the optimal configuration, it is clear that the condition of being a strict minimal configuration and the  




The densest double lattice packing of a convex polygon $K$ can be constructed in time proportional to the number of vertices by an algorithm of Mount \cite{Mount1991}. The goal of this paper is to show that this configuration is not only a local maximum
of density among double lattices, but is in fact a local maximum in a broader sense, strong extremality.










\bibliography{ngon}
\end{document}  
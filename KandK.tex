\documentclass[11pt, oneside]{article}   	% use "amsart" instead of "article" for AMSLaTeX format
\usepackage{geometry}                		% See geometry.pdf to learn the layout options. There are lots.
\geometry{letterpaper}                   		% ... or a4paper or a5paper or ... 
%\geometry{landscape}                		% Activate for for rotated page geometry
%\usepackage[parfill]{parskip}    		% Activate to begin paragraphs with an empty line rather than an indent
\usepackage{graphicx}				% Use pdf, png, jpg, or eps� with pdflatex; use eps in DVI mode
								% TeX will automatically convert eps --> pdf in pdflatex		
\usepackage{amssymb}
\usepackage{amsmath}
\usepackage{amsthm}
\usepackage [autostyle, english = american]{csquotes}

\newtheorem{definition}{Definition}[section]
\newtheorem{theorem}{Theorem}[section]
\newtheorem{conjecture}{Conjecture}[section]
\newtheorem{lemma}{Lemma}[section]
\newtheorem{conj}{Conjecture}[section]
\newtheorem{corollary}{Corollary}[section]
\newtheorem{remark}{Remark}[section]
\newtheorem{proposition}{Proposition}[section]
\newtheorem{conditions}{Conditions}[section]

%\newcommand{\defv}[1]{\textbf{\textit{#1}}}



\title{The local optimality of the double lattice packing}
\author{Yoav Kallus and W\"oden Kusner}
%\date{}							% Activate to display a given date or no date

\begin{document}
\bibliographystyle{plain}
%\maketitle
%\abstract
%This paper shows that the dense double lattice construction of Kuperberg and Kuperberg is locally optimal for polygons in the full space of packings.


\section{The construction of the densest double lattice packing}

\subsection{Double lattices}

\begin{definition}
    A chord of a convex body $K$ is a line segment whose endpoints lie on the boundary of $K$.
    A chord is an affine diameter if there is no longer chord parallel to it.
\end{definition}

\begin{definition}
    An inscribed parallelogram is a half-length parallelogram in the direction $\theta$ if
    one pair of edges is parallel with the line through the origin at an angle $\theta$ above
    the $x$-axis and their length is half the length of an affine diameter parallel to them.
\end{definition}

Note that any two half-length parallelograms in the direction $\theta$ have equal
area, and we can define that area as a function $A(\theta)$ of the direction.

\begin{definition}
    A cocompact discrete subgroup of the Euclidean group consisting of translations
    and point reflections is a \text{double lattice} if it includes at least one
    point reflection.
\end{definition}

A double lattice is generated by a lattice and a point reflection, or alternatively
by three point reflections.

\begin{theorem}[Kuperberg and Kuperberg]\label{thmkk}%strict convexity is not necessary (see Mount)
    For a convex $K$, an admissible double lattice of smallest mean area has
    mean area $4\min_\theta A(\theta)$ and is generated by reflection about
    the vertices of a half-length parallelogram.
\end{theorem}

Note that Kuperberg and Kuperberg make use of extensive parallelograms, inscribed parallelograms with edge length greater than half the affine diameter in their edge directions. Mount gives an explicit proof of the fact that it suffices to consider only the half-length parallelograms associated with affine diameters as a function of $\theta.$

For our purposes, it is illustrative to consider the algorithm used find the half-length parallelogram of minimal area not only as a method of finding the best double lattice packing, but also as a way to explore the configuration space of double lattices. More specifically, the behavior of the affine diameter is fairly well behaved but non-trivial as $\theta$ increases while the behavior of the vertices of the half parallelogram are described by an {\it interspersing property}; they are non-decreasing functions $[0,2\pi)\rightarrow [0,2\pi).$  %they are a non-decreasing function, they never back track....

We begin with a simple proposition
\begin{proposition}
there is an affine diameter of a convex polygon K in every direction that meets a vertex.
\end{proposition}

This is a consequence of the convexity K, for if a chord does not meet a vertex, it either lies within an edge or it meets two edges.  If it lies within an edge, it is not an affine diameter.  If it meets two edges, it is possible increase its length while parallel translating it in a non-decreasing direction of the cone defined by the two edges until it meets a vertex. 
%picture

From this, it is possible to determine an initial affine diameter.














\bibliography{ngon}
\end{document}  
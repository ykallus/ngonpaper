\documentclass[11pt, oneside]{article}   	% use "amsart" instead of "article" for AMSLaTeX format
\usepackage{geometry}                		% See geometry.pdf to learn the layout options. There are lots.
\geometry{letterpaper}                   		% ... or a4paper or a5paper or ... 
%\geometry{landscape}                		% Activate for for rotated page geometry
%\usepackage[parfill]{parskip}    		% Activate to begin paragraphs with an empty line rather than an indent
\usepackage{graphicx}				% Use pdf, png, jpg, or eps� with pdflatex; use eps in DVI mode
								% TeX will automatically convert eps --> pdf in pdflatex		
\usepackage{amssymb}
\usepackage{amsmath}
\usepackage{amsthm}
\usepackage{asymptote}
\usepackage [autostyle, english = american]{csquotes}
\usepackage{tikz}
\usepackage{pgfplots}
\pgfplotsset{samples=200}
\usetikzlibrary{external}

\newtheorem{definition}{Definition}[section]
\newtheorem{theorem}{Theorem}[section]
\newtheorem{conjecture}{Conjecture}[section]
\newtheorem{lemma}{Lemma}[section]
\newtheorem{conj}{Conjecture}[section]
\newtheorem{corollary}{Corollary}[section]
\newtheorem{remark}{Remark}[section]
\newtheorem{proposition}{Proposition}[section]
\newtheorem{conditions}{Conditions}[section]


%\newcommand{\defv}[1]{\textbf{\textit{#1}}}


\title{The local optimality of the double lattice packing}
\author{Yoav Kallus and W\"oden Kusner}
%\date{}							% Activate to display a given date or no date

\begin{document}
\bibliographystyle{plain}
%\maketitle
%\abstract
%This paper shows that the dense double lattice construction of Kuperberg and Kuperberg is locally optimal for polygons in the full space of packings.


\section{Introduction}
%How to view/present this.  Experimental, algorithmic or fully rigorous? Seems that full rigor is hard as it depends on certificates that are non-trivial to verify for various reasons.  
%There is an algorithm of constructing the densest double lattice packing of an arbitrary convex body in the plane.  In this paper, we extend the optimality result to a neighborhood in the space of packings.
%The story: Some of the motivation was curiosity... some existing results and explorations that could be meshed together...

This paper began as an investigation of the optimality of the double lattice packing for pentagons and heptagons.  In \cite{kuperberg1990double}, G.\ and W.\ Kuperberg describe a double lattice packing of congruent planar convex bodies with maximal density that can be constructed algorithmically\marginpar{YK: what do you mean here by ``algorithmically''? How is a nonpolygon represented computationally? \\ WK: perhaps "procedurally" for "nice enough" bodies?  }\footnote{For the case of polygons, this turns out to be linear in the number of vertices  \cite{Mount1991}.}. As an example, they construct both the densest double lattice packing for regular pentagons and the densest double lattice packing for heptagons and show that these packings have densities of $(5-\sqrt{5})/3 = 0.92131\dots$ and $0.8926\dots$ respectively. These are the current records and possibly the best \textit{general} packings of the plane by regular pentagons and by regular heptagons.

In the early part of the 2000s, there has been a significant push both theoretically and computationally to answer some of the most naive yet perplexing questions in packing \footnote{for a background on packing problems, see \cite{brass2005research} \cite{conway1999recent}\cite{groemer1963existenzsatze}.
} Along with the proof and formal verification of the Kepler conjecture \cite{hales2015formal}, a number of other results have proved to be both illuminating and frustrating (Cohn-Elkies LP bounds, Lower bounds...).  For packings by congruent anisotropic bodies, sharp results are limited mostly to the plane, where the best packings of all centrally-symmetric bodies are achieved by lattices \cite{fejes1950some}, and a series of sparce results in higher dimensions. 

For general convex bodies, the problem of finding the best packing of regular pentagons serves as a toy model for harder problems
like finding the best packing of regular tetrahedra. However, the pentagon problem is still not a tractable one. Explicit upper bounds for the packing of regular tetrahedra and octahedra are better than the trivial unity upper bound by minuscule margins \cite{gravel2011upper}. A semidefinite programming approach has been suggested by Oliveira and Vallentin \cite{mario2013computing}. Though the SDP method has not yet yielded a nontrivial upper bound for packing of tetrahedra, it has been used to obtain an upper bound of $0.98103$ on the density of regular pentagon packings. There remains a large gap between the highest density achieved for pentagon packings and this upper bound.

Even in the plane, it is an open question to find the global pessimal convex body, that is the shape that has the lowest maximum packing density.  In the class of centrally-symmetric bodies, it is Reinhardt \cite{ reinhardt1933dichteste} who conjectured that a smoothed octagon is the minimizer.  In the class of general convex bodies, it is conjectured to be the regular heptagon.  However, even though the conjectured maximum packing density of the regular heptagon is the maximal density double lattice, it has also resisted proofs to its global optimality.

The regular pentagon and heptagons are cases of special interest, and we initially sought out to investigate whether their optimal double-lattice packing can be shown
to be also optimal among a broader class of packings. We were able to show that these packings are optimal at least in some neighborhood in the space of all packings.
Furthermore, we discovered that our method can be generalized to all convex polygons. We demonstrate that, while double lattices are in general not globally
optimal, they are always at least locally optimal.

\begin{theorem}
Given a convex planar polygon, there is a double lattice packing that is locally optimal.  Modulo vertex condition... \end{theorem}

In that light, we must establish the correct notion of neighborhood.  This neighborhood should be broader than simply the Hausdorff distance between two packings.   To that end, we recall a topology on the space of sequences ambient isometries 
 
We recast the relevant portions of \cite{kuperberg1990double}.

We prove a stability result for non-linear programming problems.

We characterize the pentagon.

 We give a local parametrization of the neighborhood of the double lattice in the space of packings and give a characterization of the a correction function and show that the optimality of the local configuration problem implies the global density result.


\section{Theoretical Considerations}
\subsection{Local Stability}

%We replace density maximization with the equivalent problem of mean volume minimization.  This allows us to discard the potentially problematic density objective function, which when localized tends to depend on a large number of quantities, i.e. the areas of various pieces of elements of our packing, and replace it with localized area functions.

\begin{definition}
    Let $\Xi$ be a set of isometries. Its \textit{mean volume} is the limit
    \begin{equation}
	d(\Xi)=\lim_{r\to\infty} \frac{\mathrm{vol} B(0,r)}{|\{\xi\in\Xi : \xi(0)\in B(0,r)\}|}\text.
    \end{equation}
    The upper and lower mean volumes are the corresponding limits superior and inferior.
    We say $\Xi$ is a $(r,R)$-set if the point set $\{\xi(0):\xi\in\Xi\}$ has a packing
    radius $r$ at least $r$ and a covering radius at most $R$.
\end{definition}

We will look at packings of congruent copies of a convex body $K$. That is, every
element of the packing is given by $\xi(K)$, where $\xi$ is an isometry of Euclidean
space. It will be convenient to assume that the reference body $K$ is situated so that
its interior contains the origin.

\begin{definition}
    Let $K$ be a compact set with interior. We say that $\Xi$ is \textit{admissible} for $K$ if
    the interiors of $\xi(K)$ and $\xi'(K)$ are disjoint for any two distinct isometries $\xi,\xi'\in\Xi$.
    We say furthermore that $\Xi$ is \textit{saturated} if there is no $\xi \not\in \Xi$ such that $\Xi\cup\{\xi\}$
    is again admissible.
\end{definition}

There are $r(K)$ and $R(K)$ such that when $\Xi$ is admissible and saturated, then $\Xi$ is
a $(r(K),R(K))$-set.

\begin{definition}
    Given two $(r',R')$-sets $\Xi$ and $\Xi'$ of isometries, we define the premetric 
    \begin{equation}
	\begin{aligned}
	    \delta_R(\Xi,\Xi') = \inf_\text{enum.} \sup \{&||\xi_i^{-1}\xi_j-\xi_i'^{-1}\xi_j'||:\\ & i,j \text{ such that } ||\xi_i(0)-\xi_j(0)||<2R \text{ or } ||\xi_i'(0)-\xi_j'(0)||<2R\}\text.
	\end{aligned}
    \end{equation}
    The infimum is over all enumerations $\mathbb{N}\to\Xi$ and $\mathbb{N}\to\Xi'$.
\end{definition}

When $R>R'$, $\delta_R(\Xi,\Xi')=0$ if and only if $\xi_i = \hat\xi \xi_i'$ for some $\hat\xi\in E(n)$ and some
enumerations. Consider a body $K$. When $R>R(K)$, $\delta_R(\Xi,\Xi')$ is a metric on the space of admissible
$(r,R)$-sets up to overall isometry, which includes the saturated sets as a subset.

\begin{definition}
    We say an admissible and saturated set $\Xi$ is \textit{strongly extreme} for $K$ if it minimizes the mean volume among admissible elements in a neighborhood of $\Xi$.
\end{definition}

Note that the above definition is independent of $R$.


We note that there are packings that might be intuitively thought of as not locally optimal, but
which are strongly extreme under our definition. One example is constructed by decorating a cylinder
with a screw on its bottom base and a corrosponding screw hole boring into its top base. Consider the packing
where each cylinder is screwed into a cylinder below it, in such a way that the two cylinders
are related to each other by a translation, and the screw is not completely screwed in. This
creates a column of cylinders, copies of which we arrange in a triangular grid. Since the screw is not completely screwed in, the density of the packing can be improved by screwing it in further. Since the interlayer spacing is related to the relative rotation between
cylinders on the two layers, even an arbitrarily small consistent decrease in interlayer spacing
will cause some layers to be rotated by at least some constant angle. Such a motion is not continuous
in the topology we defined. Therefore, it is worth noting what local improvements to the 
density our results of strong extremality rule out and which are not ruled out.

However, our notion of strong extremality is stronger than previously introduced
notions of local optimality local optimality of packings. The notions of an \textit{extreme}
lattice packing and a \textit{periodic-extreme} periodic packing apply only to special
classes of packings. We show that strong extremality, which applies more generally,
implies extremality and periodic-extremality in these special classes.

\begin{theorem}
    If a lattice $\Lambda$ is strongly extreme for $K$, then $\Lambda$ is extreme for $K$ \cite{martinet2003perfect}.
\end{theorem}

\begin{theorem}
    If a periodic set $\Xi = \{ T_\mathbf{l}\xi_i : \mathbf{l}\in\Lambda, i=1,\dots,N\}$ is strongly extreme,
    then it is periodic extreme for $K$ \cite{schurmann2013strict}.
\end{theorem}

We now derive a general method for proving strong extremality which we will use in the following sections.

\begin{definition}
    Let $\Xi$ be a countable set of isometries and fix an enumeration $\Xi = \{\xi_i: i\in\mathbb{N}\}$. 
    Let $\mathcal P$ be a polyhedral complex whose underlying space is $\mathbb{R}^n$.
    For every face $F$ of $\mathcal P$, let $I_F = \{i : \xi_i(0)\in F\}$. We say $\mathcal P$ is a \textit{honeycomb} of $\Xi$ if
    each $n$-face (cell) $P$ is the convex hull of $\{\xi_i(0):i\in I_P\}$.
\end{definition}

\begin{theorem}
    Let $\Xi$ be admissible for $K$ and let $\mathcal P$ be a honeycomb of $\Xi$. For every cell $P$,
    consider the optimization problem of minimizing $f_P(\Xi_P)=\mathrm{vol}\,\mathrm{conv}_{i\in I_P}\xi'_i(0)$ over 
    the assignment of isometries $\xi'_i$, $i\in I_P$, such that this finite set is admissible.
    If $\xi'_i=\xi_i$, $i\in I_P$, is a local minimum for each cell $P$, then $\Xi$ is strongly extreme.
\end{theorem}

\begin{theorem}
    Let $g_F(\Xi_F)$ be a real-valued function over $\Xi_F=(\xi'_i)_{i\in I_F}$ for each oriented $(n-1)$-faces (ridge) of $\mathcal P$, such
    that $g_F(\Xi_F)=-g_{-F}(\Xi_F)$, where $-F$ is the orientation-reversed version of $F$. If we replace $f_P(\Xi_P)$ in the previous
    theorem with $f_P'(\Xi_P) = f_P(\Xi_P) + \sum_{F\in\partial P} g_F(\Xi_F)$, then again, if $\xi'_i=\xi_i$, $i\in I_P$, is a local minimum for each
    cell $P$, then $\Xi$ is strongly extreme.
\end{theorem}


\subsection{Double lattices}
%Intro!  In this section, we sketch the characterization of the double lattice packings as they relate to their affine diameters and half-length parallelograms.   The work of Kuperburgh and Kuperberg, the work of Mount, the stability and area criterion...


\begin{definition}
    A chord of a convex body $K$ is a line segment whose endpoints lie on the boundary of $K$.
    A chord is an affine diameter if there is no longer chord parallel to it.
\end{definition}

\begin{definition}
    An inscribed parallelogram is a half-length parallelogram in the direction $\theta$ if
    one pair of edges is parallel with the line through the origin at an angle $\theta$ above
    the $x$-axis and their length is half the length of an affine diameter parallel to them.
\end{definition}

While the half-length parallelogram in a particular direction need not be uniquely determined if $K$ is not strictly convex, any non-unique half-length parallelograms associated to a particular direction must have at least one half-length chord contained in an edge of $K$ and are equivalent by sliding motions of the chord along that edge of $K.$ Thus, any two half-length parallelograms in the direction $\theta$ have equal area, and we can define that area as a function $A(\theta)$ of the direction.  

\begin{definition}
    A cocompact discrete subgroup of the Euclidean group consisting of translations
    and point reflections is a \text{double lattice} if it includes at least one
    point reflection.
\end{definition}

A double lattice is generated by a lattice and a point reflection, or alternatively
by three point reflections.

\begin{theorem}[Kuperberg and Kuperberg]\label{thmkk}%strict convexity is not necessary (see Mount)
    For a convex $K$, an admissible double lattice of smallest mean area has
    mean area $4\min_\theta A(\theta)$ and is generated by reflection about
    the vertices of a half-length parallelogram.
\end{theorem}

%While this is 
%By a sequence of approximations, this gives that for convex bodies $K$ that are not strictly convex, there
%exists a double lattice packing of maximal density that generated a minimum area extensive parallelogram in $K$.

Note that Kuperberg and Kuperberg make use of extensive parallelograms, inscribed parallelograms with edge length greater than half the affine diameter in their edge directions, but then restrict their analysis to the set of half-length parallelograms. Mount gives an explicit proof that it suffices to consider only the half-length parallelograms associated with affine diameters as a set valued function from $\theta.$

For our purposes, it is illustrative to consider the algorithm used find the half-length parallelogram of minimal area for convex polygons, not only as a method of finding the best double lattice packing thereof, but also as a way to explore some of the configuration space of double lattice packings. More specifically, the affine diameter of a convex polygon is well behaved but with some non-trivial considerations as $\theta$ increases. The behavior of the vertices of the half-length parallelogram are described by an {\it interspersing property}; they are non-decreasing functions $[0,2\pi)\rightarrow [0,2\pi).$ %this is not best way to say this since the domain is really an isomorphism of the range, not equality...,  locally monotone ?  is there a standard way to characterize this as unsigned brouwer degree?

We begin with a simple proposition:
\begin{proposition}
there is an affine diameter of a convex polygon $K$ in every direction that meets a vertex.
\end{proposition}

This is a consequence of the convexity $K$, for if a chord does not meet a vertex, it either lies within an edge or it meets two edges.  If it lies within an edge, it is not an affine diameter.  If it meets two edges, it is possible increase its length while parallel translating it in a non-decreasing direction of the cone defined by the two edges it meets until it meets a vertex. 

From this, it is possible to determine an initial affine diameter in a particular direction $\theta$.  This is done by extending a set of parallel rays in the direction $\theta$ (or $-\theta$) from all the vertices of $K$ into the interior of $K$ and selecting the longest.  However, it does not give the initial configuration of the vertices of the half-length parallelogram or the area thereof.

\subsection{Tracking the affine diameter }
In most situations, given an initial affine diameter in a particular direction, the affine diameter can be tracked by sweeping out the ray in a clockwise direction from the initial vertex until it meets another vertex.  From there, it is possible that the affine diameter continues to extend from the original vertex or that it begins to sweep out clockwise from the opposite vertex.

The conditions for staying at the same vertex or switching to the opposite vertex are given exactly by the direction of the convex cone of the edges which the affine diameter will enter as it rotates.  The vertex from which it sweeps is that which is closer to the vertex of the cone.  In the special case where the affine diameter meets parallel edges, It is not uniquely determined, exactly because this cone is degenerate.  However, in such cases, the non-unique behavior is restricted to sliding the affine diameter along the parallel edges, and as the affine diameter continues to rotate, it will meet a new cone, although the new cone may also be degenerate. Thus, aside from the degenerate sliding configurations, the affine diameter has a choice of a {\it moving end} and a {\it fixed end} with respect to an increase in the angle $\theta.$

%figures

With regards to the double lattice packing, the affine diameter defines a column of convex bodies $K,$ and as the affine diameter is swept out from a vertex, that vertex remains the point of contact between the bodies in the column.

\subsection{Tracking the half length parallelogram}


In order to generate the densest double lattice packing, one must also find the minimal area half-length parallelogram.  This is done by moving between {\it critical angles} of the affine diameter: those angles where a vertex of the half-length parallelogram or both vertices of the affine diameter meet vertices of $K$.  It is possible to tracking the motion via a parametrization of the slopes along which the vertices must travel between certain critical angles where the the area function becomes non-analytic.  This is exactly where the moving end of the affine diameter or a vertex of the half length parallelogram meet a vertex of $K.$ Note that these also correspond to the degenerate situations not treated in Lemmas \ref{lemma:edgedg} and \ref{lemma:edgver}.


We fix  $\mathbf p_1$ and parametrize the motion of the vertices $\mathbf p_2$, $\mathbf p_3$,  $\mathbf p _5$, $\mathbf p_6$ relative to a motion of the moving end of the affine diameter $\mathbf p_4$. The moving end of the affine diameter moves linearly, as do the vertices of the half-length parallelogram, as it they are constrained to be on the edges of $K$. The length of the chords is also a linear in $t$.  Therefore, their motion is described by their initial position plus some constant velocity motion ${\mathbf v_i}$, provided that the vertices of the half-length parallelogram and the moving end of the affine diameter remain on their initial edges.  If the affine diameter or one of the vertices of the half-length parallelogram encounter a vertex, the parametrization changes to account for the new direction of motion.  In the case of the affine diameter, this may also require that the moving end become the fixed end and vice-versa.  Away from those critical angles, we have the following:

\begin{equation}
    \begin{aligned}
&{\mathbf p}_i(t) = {\mathbf p}_i^{inital} + k_i t {\mathbf v_i}\\
&\text{satisfying} \\
&\frac{{\mathbf p}_4(t)-{\mathbf p}_1}2  = \mathbf p_3(t) - \mathbf p_2(t)\\
&\text{and}\\
&\frac{{\mathbf p}_4(t)-{\mathbf p}_1}2  = \mathbf p_5(t) - \mathbf p_6(t)
    \end{aligned}
\end{equation}


This system can be solved for the rate constants $k_i$ which give conditions on the motion of the parallelogram.  We fix the affine diameter as a horizontal chord of length one, define variable motions of the point $\mathbf p_i$ at inclination $\phi_i$ and let the moving end of the affine diameter move at unit speed.  

Solving this system yields rate constants 
\begin{equation}
    \begin{aligned}
   & \alpha_1 = 0\\
&\alpha_2 = \frac{\sin \left(\phi _3-\phi _4\right)}{2 \sin \left(\phi _2-\phi _3\right)} \\
&\alpha_3 = \frac{ \sin\left(\phi _2-\phi _4\right)}{2\sin \left(\phi _2-\phi _3\right)} \\
  &\alpha_4=1\\
&\alpha_5 =  \frac{\sin \left(\phi _4-\phi _6\right)}{2\sin \left(\phi _5-\phi _6\right)}  \\
&\alpha_6 = \frac{ \sin \left(\phi _4-\phi _5\right)}{2\sin \left(\phi _5-\phi_6\right)} 
    \end{aligned}
\end{equation}


Since the critical angles where the parametrization changes can be determined, the the minimum value of $A(\theta)$ between critical angles can be solved as an optimization problem determined completely by the data at the critical angles. 

%%


It is clear that there are points which have rate constant $k_i=0$ only when there are vertices of the half-area parallelogram that are on edges parallel to the edge on which meets the moving end.  Also, the functions become singular exactly when there are two vertices of the half-parallelogram sharing an edge.  This is exactly when there is a free sliding motion that preserves area.

%behavior of the
%graphs of $A(\theta)$

%\begin{remark}
%Furthermore,  IMVT should say that ever point is met by an affine diameter as well, so every vertex is met by one.? Combining these satisfies the intuition that one may 
%\end{remark}

%continuous sweeping vs finding the minimum area intermediate configuration...
%can introduce the notion of stability here

%Having found the optimal configuration, it is clear that the condition of being a strict minimal configuration and the  


The densest double lattice packing of a convex polygon $K$ can be constructed in time proportional to the number of vertices by an algorithm of Mount \cite{Mount1991}. The goal of this paper is to show that this configuration is not only a local maximum
of density among double lattices, but is in fact a local maximum in a broader sense, strong extremality.

\subsection{General setup}

To achieve this goal, we start by describing a honeycomb associated with the double lattice. Let $K$ be a
convex polygon and let $\mathbf p_2\mathbf p_3\mathbf p _5\mathbf p_6$ be a half-length parallelogram,
such that $\mathbf p_3\mathbf p_2$ and $\mathbf p_5\mathbf p_6$ are half the length of and parallel to
the affine diameter $\mathbf p_4\mathbf p_1$. The double lattice generated by reflections about the
vertices of the parallelogram is $\Xi$ and the subgroup of translations is the lattice $\Lambda$.
Let $P=0 I_{\mathbf p_2}(0) I_{\mathbf p_6} (\mathbf p_1-\mathbf p_4)$, then $\{\xi(P):\xi\in\Xi\}$
are the cells of polyhedral complex which is a honeycomb for $\Xi$. Note that the optimization problem
of minimizing $f_{\xi(P)}$ over $\xi'_i$, $i\in I_{\xi(P)}$, is mathematically equivalent for every $\xi\in\Xi$.
Therefore, to show that Theorem X applies, it suffices to show that $\xi'_i=\xi_i$, $i\in I_P$, is a
local optimum over admissible assignments of $\xi'_i$, $i\in I_P$.

For every convex body $K$ and double lattice $\Xi$ we now have a concrete optimization problem
to solve: we wish to minimize the area of the quadrilateral $\xi'_0(0)\xi'_6(0)\xi'_1(0)\xi'_2(0)$
subject to the constraints that $\xi'_i(K)$ and $\xi'_j(K)$ do not overlap. Since the objective
and the constraints are invariant under common isometry, we may fix $\xi'_i=\xi_i$ for
one $i$. We parametrize
$\xi_i' = T_{\mathbf{r}_i}\xi_i R_{\theta_i}$, where $R_\theta$ is a rotation
by $\theta$ about the origin, and $T_{(x,y)}$ is a translation
by $\mathbf{r}_i$. Since we are only
interested in certifying that the initial configuration is a local minimum, we can replace
the constraints with ones that are equivalent in the neighborhood.

\begin{lemma}\label{lemma:edgedg}
    Let $K$ and $K'$ be two polygons that intersect at a segment, which is not
    identical with a full edge of $K$ or of $K'$. The endpoints of the segments
    are $\mathbf{x}$ a vertex of $K$ and $\mathbf{y}$ a vertex of $K'$. Let $\mathbf{y}\mathbf{y}'$
    and $\mathbf{x}\mathbf{x}'$ be the edges of $K$ and $K'$ containing the intersection.
    Let $\mathbf{x}'\mathbf{y}\mathbf{x}\mathbf{y}'$ be oriented counterclockwise
    from the point of view of the interior of $K$ (otherwise switch $K$ and $K'$).
    There is some $\epsilon>0$ such that whenever $||\xi||,||\xi'||<\epsilon$,
    then $\xi(K)$ and $\xi'(K')$ have disjoint interiors if and only if
    $\alpha(\xi(\mathbf{x})\xi(\mathbf{x}')\xi'(\mathbf{y}))\ge0$
    and $\alpha(\xi'(\mathbf{y}')\xi'(\mathbf{y})\xi(\mathbf{x}))\ge0$,
    where $\alpha$ is the signed area of the oriented triangle.
\end{lemma}

\begin{lemma}\label{lemma:edgver}
    Let $K$ and $K'$ be two polygons that intersect at a point and not at a segment.
    The intersection point $\mathbf y$ is a vertex of one polygon, which we let be $K'$,
    and sits in the relative interior of an edge $\mathbf{x}'\mathbf{x}$ of $K$,
    oriented counterclockwise from the point of view of the interior of $K$.
    There is some $\epsilon>0$ such that whenever $||\xi||,||\xi'||<\epsilon$,
    then $\xi(K)$ and $\xi'(K')$ have disjoint interiors if and only if
    $\alpha(\xi(\mathbf{x})\xi(\mathbf{x}')\xi'(\mathbf{y}))\ge0$.
\end{lemma}

Note that two cases are not treated: the case of an intersection at a point that is a vertex of both polygons
and the case of an intersection at a full edge of one or both polygons.
In the optimal double-lattice packings of the first two bodies we treat, the regular pentagon
and the regular heptagon, there are no such intersections. When we generalize
the calculation to all convex polygons, we will show that the first intersection case does not
occur when the half-length parallelogram is an isolated minimum of area, and the second
case can be treated equivalently to the case of \ref{lem:edgedg}.

\subsection{other theorems}

We will show that optimization problems we obtain fall into a convenient
form, where linear stability holds along all but one direction. Along
the direction of vanishing linear stability, the construction of
Kuperberg and Kuperberg will be shown to guaranty stability.

\begin{theorem}
Programs satisfying conditions (*) have a local maximum at 0.
\end{theorem}

\section{Calculation}

\subsection{Pentagons}

Let us fix a regular pentagon $K=\mathrm{conv} \{\mathbf{k}_i:i=0,\ldots 4\}$, where $\mathbf{k}_i = R_{2\pi i/5} (1,0)$.
In this subsection, we do all the calculations in the extension field $\mathbb{Q}(u,v)$, where $u=\cos \pi/5$ and $v=\sin \pi/5$.

One minimum-area half-length parallelogram corresponds to the affine diameter $\mathbf{p}_1\mathbf{p}_4$, where
$\mathbf{p}_1 = \mathbf{k}_0$ and $\mathbf{p}_4 = \tfrac12 (\mathbf{k}_2+\mathbf{k}_3)$.
The vertices of the parallelogram are given by
$\mathbf{p}_2=\tfrac14\mathbf k_0+\tfrac34\mathbf k_1$,
$\mathbf{p}_3=\tfrac{3-2u}4\mathbf k_1+\tfrac{1+2u}4\mathbf k_2$,
$\mathbf{p}_5=\tfrac{1+2u}4\mathbf k_3+\tfrac{3-2u}4\mathbf k_4$, and
$\mathbf{p}_6=\tfrac34\mathbf k_4+\tfrac14\mathbf k_0$.

The four pentagons that surround our primitive honeycomb cell are
$\xi_i(K)$, $i=0,1,2,6$, where $\xi_0=\mathrm{Id}$,
$\xi_1=T_{\mathbf{p}_1-\mathbf{p}_4}$,
$\xi_2=I_{\mathbf{p}_2}$,
and $\xi_6=I_{\mathbf{p}_6}$.
We are interested in showing that the assignment $\xi_i'=\xi_i$, $i=0,1,2,6$ locally
minimizes the area of the quadrilateral $\xi_0(0)\xi_6(0)\xi_1(0)\xi_2(0)$, subject
to the nonoverlap constraints. As explained in the previous section, we may
fix $\xi_1'=\xi_1$ and replace the nonoverlap constraints by signed area constraints. We
obtain the following optimization problem:

\begin{equation}\label{eq:opt-pent}
    \begin{aligned}
	\text{minimize } & f(z)=\alpha(\xi'_0(0),\xi'_1(0),\xi'_2(0))-\alpha(\xi'_0(0),\xi'_1(0)+\xi'_6(0) )\\
	\text{subj.\ to } & g_1(z)=\alpha(\xi'_0(\mathbf{k}_1),\xi'_0(\mathbf{k}_0),\xi'_2(\mathbf{k}_1))\ge0\\
	                & g_2(z)=\alpha(\xi'_2(\mathbf{k}_1),\xi'_2(\mathbf{k}_0),\xi'_0(\mathbf{k}_1))\ge0\\
	                & g_3(z)=\alpha(\xi'_0(\mathbf{k}_0),\xi'_0(\mathbf{k}_4),\xi'_6(\mathbf{k}_4))\ge0\\
	                & g_4(z)=\alpha(\xi'_6(\mathbf{k}_0),\xi'_6(\mathbf{k}_4),\xi'_0(\mathbf{k}_4))\ge0\\
	                & g_5(z)=\alpha(\xi'_1(\mathbf{k}_2),\xi'_1(\mathbf{k}_1),\xi'_2(\mathbf{k}_2))\ge0\\
	                & g_6(z)=\alpha(\xi'_2(\mathbf{k}_2),\xi'_2(\mathbf{k}_1),\xi'_1(\mathbf{k}_2))\ge0\\
	                & g_7(z)=\alpha(\xi'_1(\mathbf{k}_4),\xi'_1(\mathbf{k}_3),\xi'_6(\mathbf{k}_3))\ge0\\
	                & g_8(z)=\alpha(\xi'_6(\mathbf{k}_4),\xi'_6(\mathbf{k}_3),\xi'_1(\mathbf{k}_3))\ge0\\
	                & g_9(z)=\alpha(\xi'_1(\mathbf{k}_3),\xi'_1(\mathbf{k}_2),\xi'_0(\mathbf{k}_0))\ge0\text,
    \end{aligned}
\end{equation}
where $\xi'_i=T_{(x_i,y_i)}\xi_iR_\theta$ for $i=0,2,6$, and $\xi'_1=\xi_1$. We adopt
a condensed notation for the free variables $z=(x_0,y_0,\theta_0,x_2,y_2,\theta_2,x_6,y_6,\theta_6)$.

We consider the linearization of \ref{eq:opt-pent} around the point $z=0\in\mathbb{R}^9$. This gives a
problem of the form
\begin{equation}\label{eq:lin-pent}
    \text{minimize } c\cdot z \text{ subject to } G z\ge 0\text,
\end{equation}
where $c\in \mathbb R^9$, $G\in\mathbb{R}^{9\times9}$ and we use the linear programming notation $\ge0$
to denote a vector lying in the closed positive orthant.
We can show by direct calculation a vector $\eta>0$ lying in the open positive orthant exists such that
$c=\eta^T G$. By the fundamental theorem of linear algebra, this observation implies that $G z\ge 0$ and $c\cdot z\le0$
if and only if $G z = 0$ and $c\cdot z=0$. We can show that $\mathrm{rank} G=8$, and so the program
\ref{eq:lin-pent} is minimized exactly by the null space of $G$ and is suboptimal elsewhere in the
cone $Gz\ge0$. The null space corresponds precisely to the rearrangement given by choosing a nearby
half-length parallelogram in the double lattice construction to the minimum-area one. Let $z^{(0)}$
generate the null space, then $t_0^{(0)}=t_2^{(0)}=t_6^{(0)}=0$. We can verify directly
that $f(tz^{(0)})$ is a quadratic function of $t$ minimized at $t=0$, and that
$g_r(tz^{(0)})=0$ identically for $r=1,\ldots,9$. Indeed, perturbing the half-length
parallelogram away from the minimum-area one increases the area of the resulting cell
and maintains all the contacts.

Therefore, \ref{eq:opt-pent} satisfies all the conditions of \ref{}, and we have:

\begin{theorem}
    The optimal double-lattice packing of regular pentagons, illustrated in Figure X, is strongly extreme.
\end{theorem}

\subsection{Heptagons}

The calculation for the regular heptagon starts out the same as the calculation presented above for regular pentagons.
However, it will turn out that the linear program equivalent to \ref{eq:lin-pent} is in this case not minimized
at $z=0$, and so we will need to add auxiliary cost functions to the area as allowed for in \ref{thm:aux}.

We fix a regular heptagon $K=\mathrm{conv} \{\mathbf{k}_i:i=0,\ldots 6\}$, where $\mathbf{k}_i = R_{2\pi i/7} (1,0)$.
In this case, our calculations are performed in the extension field $\mathbb{Q}(u,v)$, where $u=\cos \pi/7$ and $v=\sin \pi/7$.
A minimum-area half-length parallelogram corresponds to the affine diameter $\mathbf{p}_1\mathbf{p}_4$, where
$\mathbf{p}_1 = \mathbf{k}_0$ and $\mathbf{p}_4 = \tfrac12 (\mathbf{k}_3+\mathbf{k}_4)$.
The vertices of the parallelogram are given by
$\mathbf{p}_2=(1-a)\mathbf k_1+a\mathbf k_2$,
$\mathbf{p}_3=(1-b)\mathbf k_2+b\mathbf k_3$,
$\mathbf{p}_5=b\mathbf k_4+(1-b)\mathbf k_5$, and
$\mathbf{p}_6=a\mathbf k_5+(1-a)\mathbf k_6$.

The four heptagons that surround our primitive honeycomb cell are again
$\xi_i(K)$, $i=0,1,2,6$, where $\xi_0=\mathrm{Id}$,
$\xi_1=T_{\mathbf{p}_1-\mathbf{p}_4}$,
$\xi_2=I_{\mathbf{p}_2}$,
and $\xi_6=I_{\mathbf{p}_6}$.
We will investigate whether the assignment $\xi_i'=\xi_i$, $i=0,1,2,6$ locally
minimizes the area of the quadrilateral $\xi_0(0)\xi_6(0)\xi_1(0)\xi_2(0)$, subject
to the nonoverlap constraints. Again, we
fix $\xi_1'=\xi_1$ and replace the nonoverlap constraints by signed area constraints. We
obtain the following optimization problem:

\begin{equation}\label{eq:opt-hept}
    \begin{aligned}
	\text{minimize } & f(z)=\alpha(\xi'_0(0),\xi'_1(0),\xi'_2(0))-\alpha(\xi'_0(0),\xi'_1(0)+\xi'_6(0) )\\
	\text{subj.\ to } & g_1(z)=\alpha(\xi'_0(\mathbf{k}_2),\xi'_0(\mathbf{k}_1),\xi'_2(\mathbf{k}_1))\ge0\\
	                & g_2(z)=\alpha(\xi'_2(\mathbf{k}_2),\xi'_2(\mathbf{k}_1),\xi'_0(\mathbf{k}_1))\ge0\\
	                & g_3(z)=\alpha(\xi'_0(\mathbf{k}_6),\xi'_0(\mathbf{k}_5),\xi'_6(\mathbf{k}_6))\ge0\\
	                & g_4(z)=\alpha(\xi'_6(\mathbf{k}_6),\xi'_6(\mathbf{k}_5),\xi'_0(\mathbf{k}_6))\ge0\\
	                & g_5(z)=\alpha(\xi'_1(\mathbf{k}_3),\xi'_1(\mathbf{k}_2),\xi'_2(\mathbf{k}_3))\ge0\\
	                & g_6(z)=\alpha(\xi'_2(\mathbf{k}_3),\xi'_2(\mathbf{k}_2),\xi'_1(\mathbf{k}_3))\ge0\\
	                & g_7(z)=\alpha(\xi'_1(\mathbf{k}_5),\xi'_1(\mathbf{k}_4),\xi'_6(\mathbf{k}_4))\ge0\\
	                & g_8(z)=\alpha(\xi'_6(\mathbf{k}_5),\xi'_6(\mathbf{k}_4),\xi'_1(\mathbf{k}_4))\ge0\\
	                & g_9(z)=\alpha(\xi'_1(\mathbf{k}_4),\xi'_1(\mathbf{k}_3),\xi'_0(\mathbf{k}_0))\ge0\text.
    \end{aligned}
\end{equation}
The linearization of \ref{eq:opt-hept} around the point $z=0\in\mathbb{R}^9$ gives
\begin{equation}\label{eq:lin-hept}
    \text{minimize } c\cdot z \text{ subject to } G z\ge 0\text.
\end{equation}
Unfortunately, \ref{eq:lin-hept} is unbounded. This can be shown by producing some
$z_u$ such that $c\cdot z_u<0$ and $G z_u\ge0$. In the dual setting, this implies
that there is no $\eta$ such that $c=\eta^T G$ and $\eta>0$.

Due to \ref{thm:aux}, we are allowed to modify the cost function $f(z)$ by adding
auxiliary functions. In order for the new problem to be locally minimized, we
need the new gradient $c'$ to lie in the cone $\{\eta^T G:\eta>0\}$. We will take
the following simple form for our modified problem
\begin{equation}\label{eq:mod-hept}
    \begin{aligned}
	\text{minimize } & f'(z) = f(z) +\sum_{r=1}^9 \mu_r g_r(z)\\
	\text{subj.\ to } & \text{same constraints as in \ref{eq:opt-hept}.}
    \end{aligned}
\end{equation}
For a cell $\xi(P)$ other than the primitive cell $P$, the modified version is the same
as \ref{eq:mod-hept}, except we replace $\xi'_i$ everywhere with $\xi'_i  = \xi\circ\xi'_i$.
Since the problem is invariant under common isometry, this is equivalent for all the cells
and it is enough to show that \ref{eq:mod-hept} is locally minimized.
Note that $\xi_2\circ\xi_0=\xi_2$ and $\xi_2\circ\xi_2=\xi_0$, so $g_1^{P}(z) = g_2^{\xi_2(P)}(z)$
and $g_2^{P}(z) = g_1^{\xi_2(P)}(z)$. Therefore, if $\mu_1=-\mu_2$, the auxiliary addition
$g_{F_2}(z)=\mu_1 g_1(z)+\mu_2 g_2(z)$ is equal in magnitude and opposite in sign for the two cells $P$ and
$\xi_2(P)$ sharing the face $F_2$,
as required in \ref{thm:aux} for an auxiliary function attached to a cell face. By similar observation,
we note that we should have $\mu_3=-\mu_4$, $\mu_5=-\mu_6$, and $\mu_7=\mu_8$. The term
$\mu_9 g_9(z)$ does not cancel with any neighboring cells, so we set $\mu_9=0$.
A choice for $\mu_r$ satisfying these condition and such that $c'$ lies in the cone
$\{\eta^T G:\eta>0\}$ exists if and only if there is some $\eta$ such that
$c=\eta^T G$ and $\eta_1+\eta_2>0$, $\eta_3+\eta_4>0$, $\eta_5+\eta_6>0$, $\eta_7+\eta_8>0$,
and $\eta_9>0$. We can show directly that such $\eta$ exists.

We now have that $G z\ge 0$ and $c'\cdot z\le0$
if and only if $G z = 0$ and $c'\cdot z=0$. The rank of the constraint matrix is again $\mathrm{rank} G=8$,
and so the program
\ref{eq:mod-hept} is minimized exactly at the one-dimensional null space of $G$.
Let $z^{(0)}$, then we can verify directly
that $f(tz^{(0)})$ is a quadratic function of $t$ minimized at $t=0$, and that
$g_r(tz^{(0)})=0$ identically for $r=1,\ldots,9$. 

Therefore, \ref{eq:mod-hept} satisfies all the conditions of \ref{}, and we have:

\begin{theorem}
    The optimal double-lattice packing of regular heptagons, illustrated in Figure X, is strongly extreme.
\end{theorem}

\subsection{General polygons}

The structure of the solution in the cases of pentagons and heptagons suggests
that it might be possible to extend the result to general convex polygons.
Assuming that the minimum-area half-length parallelogram belongs to the generic
case, the only data about the polygon that enters into the calculation are the following:
\begin{enumerate}
    \item The coordinates of vertices of the minimum-area half-length parallelogram
	and the vertices of the corresponding affine diameter. Without loss of
	generalization, we may assume the affine diameter is horizontal, of length 2,
	and bisected by the origin. The remaining data are encoded into the following
	parameters: $1+s_0$ is the height of parallelogram, $s_1$ is the mid-height
	of the parallelogram, $s_2$ and $s_3$ are the $x$-coordinates of the midpoints
	of the top and bottom sides of the parallelogram respectively.
    \item The direction of the polygon edges on which the vertices above lie, except for
	the end of the affine diameter that lies on a polygon vertex. We denote the
	inclination angles of these edges $\phi_i$, $i=2,3,4,5,6$.
    \item The distance and direction along the edge from those points to the nearest polygon vertex, that 
	is, half the length of the contact. We denote these distances $l_i$, $i=2,3,4,5,6$. For the directions,
	we will assume the directions illustrated in Figure X, but we will argue later that
	this assumption has no effect on the subsequent analysis.
\end{enumerate}

The assumption that the area of the half-length parallelogram is minimized can be written as
\begin{equation}\label{eq:stat}
    s3=
\end{equation}

As in the previous sections the objective is given by 
the area of the quadrilateral $\xi'_0(0)\xi'_6(0)\xi'_1(0)\xi'_2(0)$, and we parameterize
the search space using $z=(x_0,y_0,\theta_0,x_2,y_2,\theta_2,x_6,y_6,\theta_6)$ and $\xi'_1=\mathrm{Id}$.
The oriented triangles to be used to represent the nonoverlap constraints depend on the directions
of $l_i$, $i=2,\ldots,6$. As noted, we will assume the directions illustrated in Figure X, and
therefore the constraints are given by:

We linearize the problem to obtain a problem of the form \ref{lin-hept}.
The constraint matrix $G$ is singular, and we can obtain right and left null
space vectors $z_0$ and $\eta_0$, whose values are given in Table X.
In fact, the null spaces are spanned by these
vectors, as can be seen by noting that
$\det(G+\eta_0 z^T)/(z_0\cdot z_0) = -2048/(l_2 l_3 l_4 l_5 l_6 \sin(\phi_3-\phi2)\sin(\phi_6-\phi_5))\neq0$.
Note that the variables associated with the rotations $\theta_i$ in $z_0$ are zero,
and the assignment $z=tz_0$ corresponds to advancing the half-length parallelogram.
We therefore will have that $f(tz_0)$ is minimum at $t=0$ and that $g_r(tz_0)=0$ for all $t$.
We note also that $z_0\cdot c = 0$ and so $c$ is contained in the image of $G$.
We solve for the vector $\eta$ such that $\eta\cdot G = c$ and $\eta\cdot\eta_0=0$. We obtain the
following values:
\begin{equation}\label{eq:etas}
    \begin{aligned}
	\eta_1+\eta_2 &= -l_2\sin\phi_3\\
	\eta_3+\eta_4 &= l_3\sin\phi_2\\
	\eta_5+\eta_6 &= l_5\sin\phi_6\\
	\eta_7+\eta_8 &= -l_6\sin\phi_5\\
	\eta_8 &= -\frac{l_4}{\sin\phi_4}(1+2s_0 - \frac{\sin\phi_2\sin\phi_3}{\sin(\phi_3-\phi_2)} - \frac{\sin\phi_5\sin\phi_6}{\sin(\phi_6-\phi_5)})\text.
    \end{aligned}
\end{equation}
Since all the above are positive, we can proceed as in the case of heptagons to include auxiliary functions
that would make all the $\eta_i$'s individually positive and respect the conditions of \ref{thm:aux}.
Therefore, we have shown that the packing is strongly extreme.

To show that the directions of the contacts do not matter, the following lemma
demonstrates that the signed triangle area constraints associated with the two
directions are equivalent in their first order cones. Since we only use the
first derivatives of the constraints $g_r(z)$ in the calculation above,
it follows that it will be the same for different direction assignments.
Moreover, if the point $\mathbf{p}_i$, $i=2,3,5,$ or $6$, is a midpoint
of an edge, then the nonoverlap constraint is locally equivalent to the
union of the two constraints obtained by treating either endpoint of the
edge as the nearer one. Again, the first order cone is identical.

\begin{lemma}
    Let $\mathbf{a}$ and $\mathbf{b}$ be two distinct points, and let
    \begin{equation}
	\begin{aligned}
	    g_1(z) &= \alpha(\xi_1(\mathbf{a}),\xi_1(\mathbf{b}),\xi_2(\mathbf{b}))\\
	    g_2(z) &= \alpha(\xi_2(\mathbf{a}),\xi_2(\mathbf{b}),\xi_1(\mathbf{b}))\\
	    g'_1(z) &= \alpha(\xi_1(\mathbf{a}),\xi_1(\mathbf{b}),\xi_2(\mathbf{a}))\\
	    g'_2(z) &= \alpha(\xi_2(\mathbf{a}),\xi_2(\mathbf{b}),\xi_1(\mathbf{a}))\text,
	\end{aligned}
    \end{equation}
    where $z = (x_1,y_1,\theta_1,x_2,y_2,\theta_2)$ and $\xi_i = T_{(x_i,y_i)}R_{\mathbf{q}_i,\theta_i}$.
    Then $\{z:\nabla g_i(0)\cdot z \ge 0 \text{ for } i=1,2\}=\{z:\nabla g'_i(0)\cdot z \ge 0 \text{ for } i=1,2\}$.
\end{lemma}

\section{Formal methods}
The Wolfram System supports symbolic manipulation over extension fields via its pattern matching system.  Using it in this manner is equivalent to performing a large hand computation to determine the coefficients of the correction function, but is less prone to error.  There are no numerical estimates made in the content of the proofs. 

The case of pentagons was also formally verified using interval arithmetic.  This was possible as the final certificates for these optimization problems are computations that show some final values are bounded away from zero . That is, the condition that a vector is is in the interior of a convex cone, and that that the value of a function at zero is strictly negative, both of which can be determined by conservative error tracking at high numerical precision. For example, the Wolfram System with the interval arithmetic package supports explicit tracking of numerical intervals and error via the use of the head ${\bf Interval}$.  Then the strict positivity of the resultant interval guarantees the strict positivity of the final value which is contained in that interval.

Computations are performed in Mathematica 10.0.1.0 \cite{mathematica10}.

%$Simplify[,Assumptions->]$ head and pattern matching

\section{Slicing nonlinear programs}\label{slice}

We prove the main theorem of this section to be stated in a canonical LP form.  For the purposes of minimization of $f$, we simply consider the maximization of $-f.$  To that end, a non-linear programing problem satisfying certain conditions can be certified as locally optimal by a linear programming problem.  For the geometric problems considered, there are $a$ $priori$ configurations given by the maximal density configurations on a subspace of configuration space, namely subsets of the double lattice packings.  To produce a certificate of local optimality for this type of problem it is possible to parametrize a neighborhood of the conjectured optimal configuration and analyze the associated non-linear programming problem
$$\text{maximize }{x\in \mathbb{R}^n} f(x)\text{ subject to }g_r(x)\ge 0, r\in I$$
in a neighborhood of $0.$ An appropriate choice of parametrization allows the full non-linear program to be sliced into a one-parameter family of non-linear programs that are subordinate to the linearization of the main program at $0$. The following conditions are sufficient.  %To that end we consider a canonical parametrization where the degenerate directions are the first coordinates.

\begin{conditions}\label{assumpt}\begin{footnote}{These are the conditions that are required for the packing problems addressed. There are a number of ways they might be weakened, e.g. the condition that $E$ be 1-dimensional is not essential.}\end{footnote}\textrm{ }
\begin{enumerate}
\item \label{a finite} Let $I$ be a finite index set.
\item \label{standard} Let $e_1$ be the standard unit vector $\{1, 0,\dots,0\}$ in $\mathbb{R}^n.$
%canonical choice, dimension can be higher
\item \label{analytic} For $r$ in $I$, let $f$ and $g_r$ be analytic functions on a neighborhood of $0$.
%this can be replaced with C2, maybe C1, locally lipschitz, sub-differentiable...
\item \label{azero} Assume $f(0) = g_r(0) = 0$ for all $r$ in $I$.
%this is canonical choice
\item \label{afderiv} Let  $F(t) = \nabla f (te_1)$.
\item \label{agderiv} Let $G_r(t) = \nabla g_r(te_1)$.
\item \label{abounded} Assume the linear program
$$\text{maximize }_{x\in \mathbb{R}^n}F(0)\cdot x \textrm{ subject to }G_r(0)\cdot x \ge 0, r\in I$$
has a bounded solution and that the maximum is attained at 0.
\item \label{asolutionset} Assume that the set of solutions in $\mathbb{R}^n$ to 
$$F(0)\cdot x = 0\textrm{ subject to }G_r(0)\cdot x \ge 0, r\in I$$
is $$E := \{te_1 : t\in \mathbb{R}\}.$$
\item \label{aortho} Let $H$ be the orthogonal complement of $E$ so that $\mathbb{R}^n = E \oplus H.$ 
\item \label{anbhd} Assume there is an $\epsilon > 0$ so the functions $g_r(te_1) = 0$ for all $t\in (-\epsilon, \epsilon)$, for all $r$ in $I.$
\item \label{ahess} Assume $\frac{\partial f}{\partial t}(0) = 0$, $\frac{\partial^2 f}{\partial t^2}(0) < 0.$
%replace item above with a statement of the "degree" of the critical point?  Split into H and E, This is a degree 
\end{enumerate}
\end{conditions}


%Not sure if this should be changed to multivariable here or if the higher dimensional E cases should be reduced%

%\begin{assumptions}\label{assumpt}
%\begin{enumerate}
%\item Let $I$ be a finite index set.
%\item Let $e_i$ be the $i$-th standard unit vector.
%\item For $r$ in $I$, let $f$ and $g_r$ be analytic functions on a neighborhood of $0$.
%\item Assume $f(0) = g_r(0) = 0$ for all $r$ in $I$.
%\item Let $F_i(t) = \nabla f (t_ie_i)$.
%\item Let $G_{r,i}(t) = \nabla g_r(t_ie_i)$.
%
%\item Let $\mathbf{t} = \{t_1, \dots , t_k\}$
%\item Let $F(\mathbf{t}) = \Sigma F_i(t).$
%\item Let $G_r(\mathbf{t}) = \Sigma G_{i,r}(t).$
%
%\item Assume the linear program
%$$\text{maximize }_{x\in \mathbb{R}^n}F(0)\cdot x \textrm{ subject to }G_r(0)\cdot x \ge 0, r\in I$$
%has a bounded solution and that the maximum is attained at 0.
%\item Assume that the set of solutions in $\mathbb{R}^n$ to 
%$$F(0)\cdot x = 0\textrm{ subject to }G_r(0)\cdot x \ge 0, r\in I$$
%is $$E:= span\{e_1,\dots,e_k\}$$  %$$E := \{te_1 : t\in \mathbb{R}\}.$$ 
%\item Let $H$ be the orthogonal complement of $E$ so that $\mathbb{R}^n = E \oplus H.$ 
%\item Assume there is an $\epsilon > 0$ so the functions $g_r(te_i) = 0$ for all $t\in (-\epsilon, \epsilon)$, for all $r$ in $I$ and $i$ in ${1,\dots k}.$
%\item Assume that in $E$, $Df = 0$ and $Hf$ is negative definite.   %$\frac{\partial }{\partial t}f(0) = 0$, $\frac{\partial^2 }{\partial t^2}f(0) < 0.$
%\end{enumerate}
%\end{assumptions}

\begin{lemma}\label{lem1} Given Conditions \ref{assumpt}, the linear program 
$$\text{maximize }_{x\in H }F(0)\cdot x\textrm{ subject to }G_r(0)\cdot x \ge 0, r\in I$$
has a unique maximum at $x = 0$
\end{lemma}
\begin{proof}
By conditions \ref{abounded} and \ref{asolutionset}, the linear program
$$\text{maximize }{x\in \mathbb{R}^n}F(0)\cdot x\textrm{ subject to }G_r(0)\cdot x \ge 0, r\in I$$
is maximized exactly on $E$. The feasible set $\{x : G_r(0)\cdot x \ge 0, r\in I \textrm{ and } x \in H\}$ is a subset of the feasible set $\{x:G_r(0)\cdot x \ge 0, r\in I\}$. Thus, the program 
$$\text{maximize }_{x\in H}F(0)\cdot x\textrm{ subject to }G_r(0)\cdot x \ge 0, r\in I$$
 is maximized exactly on the non-empty intersection $$E\cap \{x :G_r(0)\cdot x\ge 0, r\in I\} \cap H = 0.$$
\end{proof}

\begin{definition}
A finitely generated cone is a subset of $\mathbb{R}^n$ which is the non-negative span of a finite set of non-zero vectors $\{v_1,\dots, v_m\}$ in $\mathbb{R}^n$, which are called the generators of the cone. 
\end{definition}
\begin{definition}
A conical linear program is a linear program with a constraint set that is a finitely generated cone.
\end{definition}

The linear programs described throughout this section are always constrained to be on the intersection of half-spaces with $0$ on the boundary. These are conical programs.

\begin{definition}
For a cone $C$, the set $C^p := \{x \in \mathbb{R}^n: v\cdot x \le 0 \textrm{ for all } v \in C\}$ is the polar cone of $C.$
\end{definition}

\begin{lemma}\label{lem2}
A conical linear program with $F\ne 0$ given by
$$\text{maximize }_{x\in \mathbb{R}^n} F\cdot x\textrm{ subject to }G_r\cdot x \ge 0, r\in I $$ 
(a) has a unique\begin{footnote}{The maximum satisfies a stronger uniqueness condition. It is stable under perturbations of $F$ and $G_k$.}\end{footnote} maximum at $x= 0$ iff $F$ is in the interior of the polar cone $C^p$ of $C= \{x : G_r\cdot x \ge 0, r\in I\}$ (b) has a bounded solution iff $F$ is in the polar cone $C^p$ of  $C =\{x : G_r\cdot x \ge 0, r\in I\}$ and attains its maximum exactly on the span of the generators $v_i$ such that $F\cdot v_i =0.$
\end{lemma}
\begin{proof}
If $F$ is in the interior of the polar cone $C^p$, then $F\cdot v_i < 0$ for all generators $v_i$.  Therefore $F\cdot x$ is uniquely maximized in $C$ at the vertex. If $F$ is on the boundary of the polar cone, then $F\cdot x$ is maximized in $C$ exactly on the span of the generators $v_i$ for which $F\cdot v_i = 0$ as $F\cdot v_j < 0$ otherwise. If $F$ is outside the polar cone, then $F\cdot v_i >0$ for some generator $v_i$. Then $F\cdot x$ is unbounded in $C$.\end{proof}

\begin{lemma}\label{lem3} Given Conditions \ref{assumpt}, there exists $\epsilon > 0$ such that for all $t$ in  $(-\epsilon,\epsilon)$, the linear program 
$$\text{maximize }_{y_t \in H}F(t)\cdot y_t$$
subject to 
$$G_r(t)\cdot y_t \ge 0, r \in I$$
has a unique maximum at $y_t = 0$.\begin{footnote}{Here $y_t$ is a dummy variable and does not depend on $t$. It is labeled $y_t$ to ease later exposition.}\end{footnote}
\end{lemma}
\begin{proof} The program for $t\in (-\epsilon,\epsilon)$, for $y_t$ in $H$, for each fixed $t$ in $(-\epsilon,\epsilon)$, for some $\epsilon> 0$, can be written as a conical program on all of $\mathbb{R}^n$ with a cone $C_t$ in $\mathbb{R}^n$ of co-dimension $\ge 1$ by introducing further constraints $e_1\cdot y_t \ge 0$ and $-e_1\cdot y_t \ge 0.$  By Lemmas \ref{lem1} and \ref{lem2}, $F(0)$ is in the polar cone of $C_0 = \{y_0: G_r(0)\cdot y_0 \ge 0, e_1\cdot y_0 \ge 0, -e_1\cdot y_0 \ge 0\}.$   As $f,g_r \in C^\omega$, the condition of  $F(t)$ being in the interior of the polar cone $C_t^p$ is open and the condition of the feasible set $C_t = \{y_t: G_r(t)\cdot y_t \ge 0, e_1\cdot y_t \ge 0, -e_1\cdot y_t \ge 0\}$ being conical is open.\begin{footnote}{The relationships between the constraint cone, the generators $v_i$ and the constraint gradients $G_k$ are nontrivial as $t$ varies, but the openness of this condition follows from the continuity of the distance function.}\end{footnote}  Therefore, by Lemma \ref{lem2} the program has a unique maximum at $y_t = 0$ for each fixed $t$ in $(-\epsilon,\epsilon)$ for some $\epsilon> 0$.\end{proof}

\begin{lemma}\label{lem4}
Given Conditions \ref{assumpt} and $\epsilon$ as in Lemma \ref{lem3}, for all $t \in (-\epsilon, \epsilon)$
there exists $\delta(t)>0$ and a cube $Q(t) \subset \mathbb{R}^n$ of side length $2\delta(t)$ such that 
$$\{(F(t) + Q(t)) \cap (\partial (C_t^p) + Q(t))\} = \emptyset.$$
\end{lemma}
\begin{proof}
This follows from Lemma \ref{lem3}, which shows $F(t)$ is in the interior of the polar cone $C_t^p$.  Then $F(t)$ and the boundary of $C_t^p$ can be separated and the existence of $Q$ is trivial.
\end{proof}

\begin{corollary}\label{cor1}
Given Conditions \ref{assumpt} and $\epsilon$ as in Lemma \ref{lem3}, for all $t \in (-\epsilon, \epsilon),$ 

$$(F(t)+\Delta) \cdot y_t \le 0$$
whenever $y_t$ satisfies
$$(G_r(t)+\Delta_r)\cdot y_t \ge 0, r\in I
\textrm{ and } 
e_1\cdot y_t \ge 0, -e_1\cdot y_t \ge 0$$
 where $\Delta$ and $\Delta_r$ are any points in the $2\delta(t)$-cube $Q(t)$ and $y_t$ is in $H$.
\end{corollary}
\begin{proof}
By Lemma \ref{lem4}, $F(t)+\Delta$ is in the interior of the polar cone $C_{t,\Delta}^p$, where $C_{t,\Delta} = \{y_t: (G_r(t)+\Delta_r)\cdot y_t \ge 0, e_1\cdot y_t \ge 0, -e_1\cdot y_t \ge 0, r\in I\}$.  
\end{proof}

\begin{lemma}\label{lem5}
Given Conditions \ref{assumpt} and $\epsilon$ as in Lemma \ref{lem3}, for all $t \in (-\epsilon, \epsilon)$, let $y_t = x-te_1 \in H$. Choose $\Delta = \Delta(y_t)$ and $\Delta_r=\Delta_r(y_t)$ in the $2\delta(t)$-cube $Q(t)$ to be the corner given by the sign of $x-te_1 = y_t.$ Then there is an $\epsilon_t$ for which 

$$(F(t)+\Delta(y_t))\cdot y_t \le 0 \implies f(x)-f(te_1) \le 0$$
and
$$(G_r(t)+\Delta_r(y_t))\cdot y_t \le 0 \implies g_r(x)-g_r(te_1) = g_r(x)\le 0$$

for all $\|y_t\| \le \epsilon_t$.
\end{lemma}
\begin{proof}
This follows from the local expansions of the nonlinear program.  By this choice of $\Delta(y_t)$ and $\Delta_r(y_t),$

$$f(x)-f(te_1)  = F(t)\cdot (x-te_1) + O(t^2) = F(t)\cdot y_t + O(t^2) $$
$$ \le  F(t)\cdot y_t + \delta(t) \|y_t\|_1 = (F(t)+\Delta(y_t))\cdot y_t $$
and using condition \ref{anbhd},
$$g_r(x) = g_r(x)-g_r(te_1) =  G_r(t)\cdot (x-te_1) + O(t^2) = G_r(t)\cdot y_t +O(t^2) $$
$$ \le  G_r(t)\cdot y_t + \delta(t) \|y_t\|_1 = (G_r(t)+\Delta_r(y_t))\cdot y_t.$$
\end{proof}

By Lemma \ref{lem4} and Corollary \ref{cor1}, for $t$ in $(-\epsilon,\epsilon)$, the program  $$\text{maximize }_{y_t \in H} (F(t) + \Delta)\cdot y_t\textrm{ subject to } (G_r + \Delta_r)\cdot y_t$$ is uniquely maximized at $y_t=0$ for any choice of $\Delta$, $\Delta_r$ in the $2\delta(t)$ cube $Q(t)$. Combined with Lemma \ref{lem5}, there is an $\epsilon_t$ neighborhood of $0$ where $f(y_t+te_1)$ is less than $f(te_1)$ on $\cup_{\Delta_r \in Q(t)}\{y_t: (G_r+\Delta_r) \cdot y_t \ge 0, r \in I, y_t \in H\}$, which contains the feasible set  $\{y_t: g_r(y_t+te_1) \ge 0, r\in I, y_t \in H\}$. Therefore the nonlinear programs $f(y_t+te_1)$ subject to $g_r(y_t+te_1)\ge 0$, $y_t \in H$, which are parameterized by $t$ in $(-\epsilon, \epsilon)$, have local maxima at $y_t=0$.  This gives the following:

\begin{theorem}\label{th1}
Given Conditions \ref{assumpt}, a fixed $t$ in $(-\epsilon, \epsilon)$ and choosing $\Delta$ and $\Delta_r$ as in Lemma \ref{lem5}, for $x$ satisfying $g_r(x)\ge 0$ for all $r$ in $I$ and $y_t = x-te_1$ in $H$, 
there exist linear programs\begin{footnote}{These programs may depend on a choice of $y_t \in H$, but $f(x)$ is always less then $f(te_1)$ by Lemma \ref{lem5}.}\end{footnote} $$\text{maximize }_{y_t \in H}(F(t)+\Delta(y_t))\cdot y_t \textrm{ subject to }  (G_r(t)+\Delta_r(y_t))\cdot y_t \ge 0$$
that give solutions to the nonlinear programs $$\text{maximize }_{x \in H+te_1} f(x)  \textrm{ subject to }   g_r(x)\ge 0$$
in an $\epsilon_t$ neighborhood of $te_1$ in $H+te_1.$ \qed
\end{theorem}

By choice of a sufficiently small $\epsilon$ and a minimal\begin{footnote}{This exists by a compactness argument.}\end{footnote} non-zero $\epsilon_t$, Theorem \ref{th1} gives an open neighborhood of $0$ in which the maximum value of the original nonlinear program occurs on $E$. The conditions for the first and second $t$-derivatives at $0$ shows $0$ to be a local maximum for the nonlinear program
 $$\text{maximize }_{x\in\mathbb{R}^n}f(x) \textrm{ subject to }  g_r(x) \ge 0.$$
%can remove the condition on derivatives?  
\begin{theorem}
A nonlinear program satisfying Conditions \ref{assumpt} has an isolated local maximum at 0 with f(0) = 0. \qed
\end{theorem}

\bibliography{ngon}



\section{Figures}
\begin{figure}\begin{center}
\begin{asy}
import cseblack;
import olympiad;
usepackage("amssymb");
size(250);
pair K0=D("\mathbf{k}_0,\mathbf{p}_1",(1,0),E);
pair K1=D("\mathbf{k}_1",rotate(360./5.)*K0,NE);
pair K2=D("\mathbf{k}_2",rotate(360./5.)*K1,NW);
pair K3=D("\mathbf{k}_3",rotate(360./5.)*K2,SW);
pair K4=D("\mathbf{k}_4",rotate(360./5.)*K3,SE);
path KK = D(K0--K1--K2--K3--K4--cycle,linewidth(1));

real u = cos(pi/5.);
pair P1=K0;
pair P2=D("\mathbf{p}_2",(K0+3.*K1)/4.,NE);
pair P3=D("\mathbf{p}_3",((3.-2.*u)*K1+(1.+2.*u)*K2)/4.,NW);
pair P4=D("\mathbf{p}_4",(2.*K2+2.*K3)/4.,SW);
pair P5=D("\mathbf{p}_5",((3.-2.*u)*K4+(1.+2.*u)*K3)/4.,SW);
pair P6=D("\mathbf{p}_6",(K0+3.*K4)/4.,SE);
path PP = P2--P3--P5--P6--cycle;
fill(PP,lightgray);
D(PP);
D(P1--P4);

pair V2 = 1.*(K1-P2);
pair V3 = (1./sqrt(5.))*(K2-P3);
pair V4 = 1.*(K3-P4);
pair V5 = (sqrt(5.)-2.)*(K4-P5);
pair V6 = (1./3.)*(K0-P6);
real t = -0.6;
path PP = D((P2+t*V2)--(P3+t*V3)--(P5+t*V5)--(P6+t*V6)--cycle,dashed);
D( P1--(P4+t*V4),dashed);

MC ("y",D((P4-(0.1,0))--(P4+t*V4-(0.1,0)),Arrow),0.5,W);
\end{asy}
\scalebox{.8}{
    \tikzsetnextfilename{pentplot}
    \begin{tikzpicture}
	\begin{axis}[
	ylabel style={rotate=-90},
		xlabel={$y$}, ylabel={$A$},
		xmin=-1., xmax=1.,
		ymin=1.25, ymax=1.4,
		%y label style={at={(0.05,0.5)}},
		%xticklabels = {$2$,$5$,$10$,$20$,$50$},
		%xtick={2,5,10,20,50}
	    ]
	    \addplot[] {1.2903580504417251 + 0.10153740507278321*x^2};
	\end{axis}
    \end{tikzpicture}
}
\caption{\label{fig:pent}
Half-length parallelograms in the regular pentagon.}
\end{center}\end{figure}

\begin{figure}\begin{center}
\begin{asy}
import cseblack;
import olympiad;
usepackage("amssymb");
size(250);
pair K0=D("\mathbf{k}_0,\mathbf{p}_1",(1,0),E);
pair K1=D("\mathbf{k}_1",rotate(360./7.)*K0,NE);
pair K2=D("\mathbf{k}_2",rotate(360./7.)*K1,N);
pair K3=D("\mathbf{k}_3",rotate(360./7.)*K2,NW);
pair K4=D("\mathbf{k}_4",rotate(360./7.)*K3,SW);
pair K5=D("\mathbf{k}_5",rotate(360./7.)*K4,S);
pair K6=D("\mathbf{k}_6",rotate(360./7.)*K5,SE);
path KK = D(K0--K1--K2--K3--K4--K5--K6--cycle,linewidth(1));

pair P1=(1., 0.);
pair P2=D("\mathbf{p}_2",(0.516460816041133, 0.806260150051493),N);
pair P3=D("\mathbf{p}_3",(-0.4340236179100764, 0.8062601500514931),NW);
pair P4=D("\mathbf{p}_4",(-0.9009688679024191, 0.),SW);
pair P5=D("\mathbf{p}_5",(-0.4340236179100764, -0.8062601500514931),SW);
pair P6=D("\mathbf{p}_6",(0.516460816041133, -0.806260150051493),S);
path PP = P2--P3--P5--P6--cycle;
fill(PP,lightgray);
D(PP);
D(P1--P4);
pair P4x = P4;

pair P1=(1., 0.);
pair P2=(0.36670303301673796, -0.8404413867659472);
pair P3=(-0.5837814009344715, -0.686832303311513);
pair P4=(-0.9009688679024191, 0.3072181669088686);
pair P5=(-0.31174490092936674, 0.9037741728702919);
pair P6=(0.6387395330218427, 0.7501650894158576);
path PP = P2--P3--P5--P6--cycle;
D(PP,dashed);
D(P1--P4,dashed);

MC ("y",D((P4x-(0.1,0))--(P4-(0.1,0)),Arrow),0.5,W);
\end{asy}
\scalebox{.8}{
    \tikzsetnextfilename{pentplot}
    \begin{tikzpicture}
	\begin{axis}[
	ylabel style={rotate=-90},
		xlabel={$y$}, ylabel={$A$},
		xmin=-1., xmax=1.,
		ymin=1.52, ymax=1.58,
	    ]
	    \addplot[domain=-0.506040792565066:0.506040792565066] {1.5326754446782211 + 0.09176757534725741*x^2};
	    \addplot[domain=0.506040792565066:1] {1.5848482175212668 - 0.08816765628922468*x + 0.06225952189759027*x^2};
	    \addplot[domain=-1:-0.506040792565066] {1.5848482175212668 + 0.08816765628922468*x + 0.06225952189759027*x^2};
	\end{axis}
    \end{tikzpicture}
}
\caption{\label{fig:hept}
Half-length parallelograms in the regular heptagon. The dashed configuration is the local minimum at $y>0$.}
\end{center}\end{figure}

\begin{figure}\begin{center}
\begin{asy}
import cseblack;
import olympiad;
usepackage("amssymb");
size(250);
pair K0=D("\mathbf{k}_0,\mathbf{p}_1",(1,0),E);
pair K1=D("\mathbf{k}_1",rotate(360./9.)*K0,NE);
pair K2=D("\mathbf{k}_2",rotate(360./9.)*K1,N);
pair K3=D("\mathbf{k}_3",rotate(360./9.)*K2,NW);
pair K4=D("\mathbf{k}_4",rotate(360./9.)*K3,W);
pair K5=D("\mathbf{k}_5",rotate(360./9.)*K4,W);
pair K6=D("\mathbf{k}_6",rotate(360./9.)*K5,SW);
pair K7=D("\mathbf{k}_7",rotate(360./9.)*K6,S);
pair K8=D("\mathbf{k}_8",rotate(360./9.)*K7,SE);
path KK = D(K0--K1--K2--K3--K4--K5--K6--K7--K8--cycle,linewidth(1));

pair P1=(1., 0.);
pair P2=D("\mathbf{p}_6",(0.3706844903100875, -0.87104878486755),SE);
pair P3=D("\mathbf{p}_5",(-0.5991618200828668, -0.7478489484525013),SW);
pair P4=D("\mathbf{p}_4",(-0.9396926207859084, 0.24639967283009717),NE);
pair P5=D("\mathbf{p}_3",(-0.4058322193092865, 0.882629724233649),N);
pair P6=D("\mathbf{p}_2",(0.5640140910836678, 0.7594298878186004),NE);
path PP = P2--P3--P5--P6--cycle;
fill(PP,lightgray);
D(PP);
D(P1--P4);
pair P4x = P4;

pair P1=(1., 0.);
pair P2=(0.4403241757329789, 0.8308422937423627);
pair P3=(-0.5295221346599754, 0.8308422937423627);
pair P4=(-0.9396926207859084, 0.);
pair P5=(-0.5295221346599754, -0.8308422937423627);
pair P6=(0.4403241757329789, -0.8308422937423627);
path PP = P2--P3--P5--P6--cycle;
D(PP,dashed);
D(P1--P4,dashed);

MC ("y",D((P4-(0.1,0))--(P4x-(0.1,0)),Arrow),0.5,W);
\end{asy}
\scalebox{.8}{
    \tikzsetnextfilename{pentplot}
    \begin{tikzpicture}
	\begin{axis}[
	ylabel style={rotate=-90},
		xlabel={$y$}, ylabel={$A$},
		xmin=-1., xmax=1.,
		ymin=1.60, ymax=1.62,
		ytick={1.60,1.61,1.62}
	    ]
	    \addplot[domain=-0.30540728933227923:0.30540728933227923] {1.6115786662088993+ 0.033061308882837155*x^2};
	    \addplot[domain=0.30540728933227923:1] {1.633850356689106 - 0.0797264910024892*x + 0.05533299936304392*x^2};
	    \addplot[domain=-1:-0.30540728933227923] {1.633850356689106 + 0.0797264910024892*x + 0.05533299936304392*x^2};
	\end{axis}
    \end{tikzpicture}
}
\caption{\label{fig:hept}
Half-length parallelograms in the regular 9-gon. The minimum at $y=0$ is not the global minimum.}
\end{center}\end{figure}

\end{document}  

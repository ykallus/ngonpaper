\documentclass[11pt, oneside]{article}   	% use "amsart" instead of "article" for AMSLaTeX format
\usepackage{geometry}                		% See geometry.pdf to learn the layout options. There are lots.
\geometry{letterpaper}                   		% ... or a4paper or a5paper or ... 
%\geometry{landscape}                		% Activate for for rotated page geometry
%\usepackage[parfill]{parskip}    		% Activate to begin paragraphs with an empty line rather than an indent
\usepackage{graphicx}				% Use pdf, png, jpg, or eps� with pdflatex; use eps in DVI mode
								% TeX will automatically convert eps --> pdf in pdflatex		
\usepackage{amssymb}
\usepackage{amsmath}
\usepackage{amsthm}
\usepackage [autostyle, english = american]{csquotes}

\newtheorem{definition}{Definition}[section]
\newtheorem{theorem}{Theorem}[section]
\newtheorem{conjecture}{Conjecture}[section]
\newtheorem{lemma}{Lemma}[section]
\newtheorem{conj}{Conjecture}[section]
\newtheorem{corollary}{Corollary}[section]
\newtheorem{remark}{Remark}[section]
\newtheorem{proposition}{Proposition}[section]
\newtheorem{conditions}{Conditions}[section]

%\newcommand{\defv}[1]{\textbf{\textit{#1}}}


\title{The local optimality of the double lattice packing}
\author{Yoav Kallus and W\"oden Kusner}
%\date{}							% Activate to display a given date or no date

\begin{document}
\bibliographystyle{plain}
%\maketitle
%\abstract
%This paper shows that the dense double lattice construction of Kuperberg and Kuperberg is locally optimal for polygons in the full space of packings.


\section{Introduction}

%How to view/present this.  Experimental, algorithmic or fully rigorous? Seems that full rigor is hard as it depends on certificates that are non-trivial to verify for various reasons.  

references for packing
\cite{brass2005research} %find better references for intro to packing
\cite{conway1999recent}
references for anisotropic packing problems
The packing problem for centrally symmetric bodies in the plane is mostly understood.
\cite{fejes1950some}
\cite{kuperberg1990double}
\cite{gravel2011upper}
\cite{mario2013computing}
references for mathematica
\cite{mathematica10}
\cite{keiper1995interval} %outdated
references for linear programming

We prove the Kuperberg and Kuperberg double lattice construction for polygons is an isolated local maximum for density in the full space of packings.  That is, the density of nearby packings are of strictly lower density.



%There is an algorithm of constructing the densest double lattice packing of an arbitrary convex body in the plane.  In this paper, we extend the optimality result to a neighborhood in the space of packings.

The structure of the paper is as follows.  

We review the required results from [] lattices, packing...

We review Kuperberg and Kuperberg

 We give a local parametrization of the neighborhood of the double lattice in the space of packings and give a characterization of the a correction function and show that the optimality of the local configuration problem implies the global density result.

We prove a stability result for non-linear programming problems of the type described.

the density function can be replaced with a function of the correct type and which implies the density result.

\section{Theoretical Considerations}
\subsection{Local Stability}

\begin{definition}
    Let $\Xi$ be a set of isometries. Its \textit{mean volume} is the limit
    \begin{equation}
	d(\Xi)=\lim_{r\to\infty} \frac{\mathrm{vol} B(0,r)}{|\{\xi\in\Xi : \xi(0)\in B(0,r)\}|}\text.
    \end{equation}
    The upper and lower mean volumes are the corresponding limits superior and inferior.
    We say $\Xi$ is a $(r,R)$-set if the point set $\{\xi(0):\xi\in\Xi\}$ has a packing
    radius $r$ at least $r$ and a covering radius at most $R$.
\end{definition}

\begin{definition}
    Let $K$ be a compact set with interior. We say that $\Xi$ is \textit{admissible} for $K$ if
    the interiors of $\xi(K)$ and $\xi'(K)$ are disjoint for any two distinct isometries $\xi,\xi'\in\Xi$.
    We say furthermore that $\Xi$ is \textit{saturated} if there is no $\xi$ such that $\Xi\cup\{\xi\}$
    is again admissible.
\end{definition}

There are $r(K)$ and $R(K)$ such that when $\Xi$ is admissible and saturated, then $\Xi$ is
a $(r(K),R(K))$-set.

\begin{definition}
    Given two $(r',R')$-sets $\Xi$ and $\Xi'$ of isometries, we define the premetric 
    \begin{equation}
	\begin{aligned}
	    \delta_R(\Xi,\Xi') = \inf_\text{enum.} \sup \{&||\xi_i^{-1}\xi_j-\xi_i'^{-1}\xi_j'||:\\ & i,j \text{ such that } ||\xi_i(0)-\xi_j(0)||<2R \text{ or } ||\xi_i'(0)-\xi_j'(0)||<2R\}\text.
	\end{aligned}
    \end{equation}
    The infimum is over all enumerations $\mathbb{N}\to\Xi$ and $\mathbb{N}\to\Xi$.
\end{definition}

When $R>R'$, $\delta_R(\Xi,\Xi')=0$ if and only if $\xi_i = \phi \xi_i'$ for some $\phi\in E(n)$ and some
enumerations. Consider a body $K$. When $R>R(K)$, $\delta_R(\Xi,\Xi')$ is a metric on the space of admissible
$(r,R)$-sets up to overall isometry, which includes the saturated sets as a subset.

\begin{definition}
    We say an admissible and saturated set $\Xi$ is \textit{strongly extreme} for $K$ if it minimizes the mean volume among admissible elements in a neighborhood of $\Xi$.
\end{definition}

Note that the above definition is independent of $R$.

Our notion of strong extremality is meant to generalize weaker notions of local optimality
for packing. Specifically, these notions apply to lattices and periodic packings.

\begin{remark}
    If a lattice $\Lambda$ is strongly extreme for $K$, then $\Lambda$ is extreme for $K$ \cite{Martinet2003}.
\end{remark}
\begin{remark}
    If a periodic set $\Xi = \{ T_\mathbf{l}\xi_i : \mathbf{l}\in\Lambda, i=1,\dots,N\}$ is strongly extreme,
    then it is periodic extreme for $K$ \cite{Schurmann2013}.
\end{remark}

\begin{definition}
    Let $\Xi$ be a countable set of isometries and fix an enumeration $\Xi = \{\xi_i: i\in\mathbb{N}\}$. 
    Let $\mathcal P$ be a polyhedral complex whose underlying space is $\mathbb{R}^n$.
    For every face $F$ of $\mathcal P$, let $I_F = \{i : \xi_i(0)\in F\}$. We say $\mathcal P$ is a \textit{honeycomb} of $\Xi$ if
    each $n$-face (cell) $P$ is the convex hull of $\{\xi_i(0):i\in I_P\}$.
\end{definition}

\begin{theorem}
    Let $\Xi$ be admissible for $K$ and let $\mathcal P$ be a honeycomb of $\Xi$. For every cell $P$,
    consider the optimization problem of minimizing $f_P(\Xi_P)=\mathrm{vol}\,\mathrm{conv}_{i\in I_P}\xi'_i(0)$ over 
    the assignment of isometries $\xi'_i$, $i\in I_P$, such that this finite set is admissible.
    If $\xi'_i=\xi_i$, $i\in I_P$, is a local minimum for each cell $P$, then $\Xi$ is strongly extreme.
\end{theorem}

\begin{theorem}
    Let $g_F(\Xi_F)$ be a real-valued function over $\Xi_F=(\xi'_i)_{i\in I_F}$ for each oriented $(n-1)$-faces (ridge) of $\mathcal P$, such
    that $g_F(\Xi_F)=-g_{-F}(\Xi_F)$, where $-F$ is the orientation-reversed version of $F$. If we replace $f_P(\Xi_P)$ in the previous
    theorem with $f_P'(\Xi_P) = f_P(\Xi_P) + \sum_{F\in\partial P} g_F(\Xi_F)$, then again, if $\xi'_i=\xi_i$, $i\in I_P$, is a local minimum for each
    cell $P$, then $\Xi$ is strongly extreme.
\end{theorem}


\subsubsection{Summary of \cite[\S 2]{kuperberg1990double} on double lattices.  }

\begin{definition}
    A chord of a convex body $K$ is a line segment whose endpoints lie on the boundary of $K$.
    A chord is an affine diameter if there is no longer chord parallel to it.
\end{definition}

\begin{definition}
    An inscribed parallelogram is a half-length parallelogram in the direction $\theta$ if
    one pair of edges is parallel with the line through the origin at an angle $\theta$ above
    the $x$-axis and their length is half the length of an affine diameter parallel to them.
\end{definition}

Note that any two half-length parallelograms in the direction $\theta$ have equal
area, and we can define that area as a function $A(\theta)$ of the direction.

\begin{definition}
    A cocompact discrete subgroup of the Euclidean group consisting of translations
    and point reflections is a \text{double lattice} if it includes at least one
    point reflection.
\end{definition}

A double lattice is generated by a lattice and a point reflection, or alternatively
by three point reflections.

\begin{theorem}[Kuperberg and Kuperberg]\label{thmkk}%strict convexity is not necessary (see Mount)
    For a convex $K$, an admissible double lattice of smallest mean area has
    mean area $4\min_\theta A(\theta)$ and is generated by reflection about
    the vertices of a half-length parallelogram.
\end{theorem}

%By a sequence of approximations, this gives that for convex bodies $K$ that are not strictly convex, there
%exists a double lattice packing of maximal density that generated a minimum area extensive parallelogram in $K$.

The densest double lattice packing of a convex polygon $K$ can be constructed in time proportional to the number of vertices
by an algorithm of Mount \cite{Mount1991}. The goal of this paper is to show that this configuration is not only a local maximum
of density among double lattices, but is in fact a local maximum in a broader sense, strong extremality.

To achieve this goal, we start by describing a honeycomb associated with the double lattice. Let $K$ be a
convex polygon and let $\mathbf p_2\mathbf p_3\mathbf p _5\mathbf p_6$ be a half-length parallelogram,
such that $\mathbf p_3\mathbf p_2$ and $\mathbf p_5\mathbf p_6$ are half the length of and parallel to
the affine diameter $\mathbf p_4\mathbf p_1$. The double lattice generated by reflections about the
vertices of the parallelogram is $\Xi$ and the subgroup of translations is the lattice $\Lambda$.
Let $P=0 I_{\mathbf p_2}(0) I_{\mathbf p_6} (\mathbf p_1-\mathbf p_4)$, then $\{\xi(P):\xi\in\Xi\}$
are the cells of polyhedral complex which is a honeycomb for $\Xi$. Note that the optimization problem
of minimizing $f_{\xi(P)}$ over $\xi'_i$, $i\in I_{\xi(P)}$, is mathematically equivalent for every $\xi\in\Xi$.
Therefore, to show that Theorem X applies, it suffices to show that $\xi'_i=\xi_i$, $i\in I_P$, is a
local optimum over admissible assignments of $\xi'_i$, $i\in I_P$.

For every convex body $K$ and double lattice $\Xi$ we now have a concrete optimization problem
to solve: we wish to minimize the area of the quadrilateral $\xi'_0(0)\xi'_6(0)\xi'_1(0)\xi'_2(0)$
subject to the constraints that $\xi'_i(K)$ and $\xi'_j(K)$ do not overlap. Since we are only
interested in certifying that the initial configuration is a local minimum, we can replace
the constraints with ones that are equivalent in the neighborhood.

\begin{lemma}
    Let $K$ and $K'$ be two polygons that intersect at a segment. The endpoints of the segments
    are $\mathbf{x}$ a vertex of $K$ and $\mathbf{y}$ a vertex of $K'$. Let $\mathbf{y}\mathbf{y}'$
    and $\mathbf{x}\mathbf{x}'$ be the edges of $K$ and $K'$ containing the intersection.
    Let $\mathbf{x}'\mathbf{y}\mathbf{x}\mathbf{y}'$ be oriented counterclockwise
    from the point of view of the interior of $K$ (otherwise switch $K$ and $K'$).
    There is some $\epsilon>0$ such that whenever $||\xi||,||\xi'||<\epsilon$,
    then $\xi(K)$ and $\xi'(K')$ have disjoint interiors if and only if
    $\alpha(\xi(\mathbf{x})\xi(\mathbf{x}')\xi'(\mathbf{y}))\ge0$
    and $\alpha(\xi'(\mathbf{y}')\xi'(\mathbf{y})\xi(\mathbf{x}))\ge0$,
    where $\alpha$ is the signed area of the oriented triangle.
\end{lemma}

\begin{lemma}
    Let $K$ and $K'$ be two polygons that intersect at a point and not at a segment.
    The intersection point $\mathbf y$ is a vertex of one polygon, which we let be $K'$,
    and sits in the relative interior of an edge $\mathbf{x}'\mathbf{x}$ of $K$,
    oriented counter clockwise from the point of view of the interior of $K$.
    There is some $\epsilon>0$ such that whenever $||\xi||,||\xi'||<\epsilon$,
    then $\xi(K)$ and $\xi'(K')$ have disjoint interiors if and only if
    $\alpha(\xi(\mathbf{x})\xi(\mathbf{x}')\xi'(\mathbf{y}))\ge0$.
\end{lemma}

Note that the case of an intersection at a point that is a vertex of both polygons
is not treated. We exclude this case in the following lemma.

\begin{lemma}
    Let $K$ be a convex polygon and let $\theta$ be such that $A(\theta)$
    is minimal, then there is a half-length parallelogram in the
    direction $\theta$ that shares no vertex with $K$ and an affine
    diameter with no more than one vertex shared with $K$.
\end{lemma}

\subsubsection{other theorems}

\begin{theorem}
Programs satisfying conditions (*) have a local maximum at 0.
\end{theorem}

\section{Calculation}

\subsection{Pentagons}

rewrite in terms of section 1

In the case of packings by regular pentagons, one can consider the configuration space of four pentagons with respect to the density function taken with respect to the Delaunay triangulation and parametrized by ....  This can be shown to satisfy the conditions of Section \ref{}....


\subsubsection{ exhibit the construction given by \ref{thmkk}}

\subsubsection{parametrize neighborhood of four pentagons in correct configuration }

the parametrization is important, should be consistent with the later sections.  Can this be hidden in the code, and to just state that there exists a parametrization that satisfies the linear program theorem?  Still need a sketch here. 

\subsubsection{constrained optimization problem}

\subsubsection{density function}

\subsubsection{verification}


\subsection{Heptagons}

Outline of the method for heptagons.

For heptagons, the cost function is non-trivial.  This is because there is a motion in the configuration space of four heptagons that increases the double Delauney density.  

(include figure)



\subsubsection{ find the construction given by Theorem \ref{thmkk}.}

\subsubsection{parametrize neighborhood of four heptagons in correct configuration}
include figures

\subsubsection{constrained optimization problem}




\subsubsection{the cost functions }

Construct a cost function that satisfies the condition of correction theorems and makes the program satisfy the conditions of the LP theorems.


YOU GET THE COST FUNCTION FROM THE PROGRAM, AND IT SATISFIES THE PROPERTIES OF SECTION 1....

This area could be confusing as the objective functions become fairly complicated.  Need to justify replacing objective functions with modifications.  i.e., 

Sketch: Minimize the area of the Delaunay triangles.  If there is no nearby configuration that decreases area done. (can we increase area and still increase density? possibly locally, so need to trade between nearby double triangles.  But by symmetry, introduce a cost function between the "outer" heptagons, i.e. penalize rotation in opposite directions.)
In general, average area of double triangles is minimized implies average density is maximized.

The function optimized in this case is the (modified) area of the Delaunay triangles. 

%$U$ is the area of the double Delaunay triangle
%
%$$Ux = U + (1/2 + u1^2) V[[3]] - (1/2 + u1^2) V[[4]] + (1/2 + u1^2) V[[
%    8]] - (1/2 + u1^2) V[[9]]$$
    
          
%1) modification given was constructed to satisfy XYZ to give the local optimality and satisfy certain properties
%2) Justify modification via it's properties, not via what the actually modification is.


% \begin{lemma}
% an objective function satisfying properties XYZ has an isolated max at 0 and satisfying properties ABC also maximizes the density function.
% \end{lemma} 
% 
% \begin{proof}
% XYZ follows from theorem.
%
% ABC follows from ...  geometric sketch argument should be replaced by an analysis argument.
%\qed \end{proof}
%
%
%\begin{lemma}
%the construction of 
%\end{lemma}




\subsection{general (2n+1)-gons}
failure in the case of the enneagon, the construction of K+K is problematic.  This requires additional analysis as 

four polygons with corrected area function and an SL(2,R) motion hessian?  The SL2 motion is required for the solution part, i.e. finding the initial configuration for regular polygons.  Check this makes sense.

\section{Formal methods}%Need good documentation for mathematica packages and for the code to be supplied.
%Computations are performed in Mathematica 10.0.1.0
%
%\subsection{interval arithmetic}
%The final certificates for these optimization problems are computations that show some final values is strictly positive.  This can be performed numerically given proper precision and accuracy control with regards to rounding error.  For example, the Wolfram System supports explicit tracking of numerical intervals and error via the use of the$Interval[]$ head.  Then the strict positivity of of the resultant interval guarantees the strict positivity of the final value which is contained in that interval.


\subsection{symbolic computation in an extension field}
The Wolfram System supports symbolic computation over extension fields via it's pattern matching system.


%$Simplify[,Assumptions->]$ head and pattern matching



\section{Slicing nonlinear programs}\label{slice}

A non-linear programing problem satisfying certain conditions can be certified as locally optimal by a linear programming problem.  For the geometric problems considered, there are $a$ $priori$ configurations given by the maximal density configurations on a subspace of configuration space, namely subsets of the double lattice packings.  To produce a certificate of local optimality for this type of problem it is possible to parametrize a neighborhood of the conjectured optimal configuration and analyze the associated non-linear programming problem
$$\max_{x\in \mathbb{R}^n} f(x)\textrm{ subject to }g_r(x)\ge 0, r\in I$$
in a neighborhood of $0.$ An appropriate choice of parametrization allows the full non-linear program to be sliced into a one-parameter family of non-linear programs that are subordinate to the linearization of the main program at $0$. The following conditions are required.

\begin{conditions}\label{assumpt}\begin{footnote}{These are the conditions that are required for the packing problems addressed. There are a number of ways they might be weakened, e.g. the condition that $E$ be 1-dimensional is not essential.}\end{footnote}\textrm{ }
\begin{enumerate}
\item \label{a finite} Let $I$ be a finite index set.
\item \label{standard} Let $e_1$ be the standard unit vector $\{1, 0,\dots,0\}$ in $\mathbb{R}^n.$
\item \label{analytic} For $r$ in $I$, let $f$ and $g_r$ be analytic functions on a neighborhood of $0$.
\item \label{azero} Assume $f(0) = g_r(0) = 0$ for all $r$ in $I$.
\item \label{afderiv} Let  $F(t) = \nabla f (te_1)$.
\item \label{agderiv} Let $G_r(t) = \nabla g_r(te_1)$.
\item \label{abounded} Assume the linear program
$$\max_{x\in \mathbb{R}^n}F(0)\cdot x \textrm{ subject to }G_r(0)\cdot x \ge 0, r\in I$$
has a bounded solution and that the maximum is attained at 0.
\item \label{asolutionset} Assume that the set of solutions in $\mathbb{R}^n$ to 
$$F(0)\cdot x = 0\textrm{ subject to }G_r(0)\cdot x \ge 0, r\in I$$
is $$E := \{te_1 : t\in \mathbb{R}\}.$$
\item \label{aortho} Let $H$ be the orthogonal complement of $E$ so that $\mathbb{R}^n = E \oplus H.$ 
\item \label{anbhd} Assume there is an $\epsilon > 0$ so the functions $g_r(te_1) = 0$ for all $t\in (-\epsilon, \epsilon)$, for all $r$ in $I.$
\item \label{ahess} Assume $\frac{\partial }{\partial t}f(0) = 0$, $\frac{\partial^2 }{\partial t^2}f(0) < 0.$
\end{enumerate}
\end{conditions}


%Not sure if this should be changed to multivariable here or if the higher dimensional E cases should be reduced%

%\begin{assumptions}\label{assumpt}
%\begin{enumerate}
%\item Let $I$ be a finite index set.
%\item Let $e_i$ be the $i$-th standard unit vector.
%\item For $r$ in $I$, let $f$ and $g_r$ be analytic functions on a neighborhood of $0$.
%\item Assume $f(0) = g_r(0) = 0$ for all $r$ in $I$.
%\item Let $F_i(t) = \nabla f (t_ie_i)$.
%\item Let $G_{r,i}(t) = \nabla g_r(t_ie_i)$.
%
%\item Let $\mathbf{t} = \{t_1, \dots , t_k\}$
%\item Let $F(\mathbf{t}) = \Sigma F_i(t).$
%\item Let $G_r(\mathbf{t}) = \Sigma G_{i,r}(t).$
%
%\item Assume the linear program
%$$\max_{x\in \mathbb{R}^n}F(0)\cdot x \textrm{ subject to }G_r(0)\cdot x \ge 0, r\in I$$
%has a bounded solution and that the maximum is attained at 0.
%\item Assume that the set of solutions in $\mathbb{R}^n$ to 
%$$F(0)\cdot x = 0\textrm{ subject to }G_r(0)\cdot x \ge 0, r\in I$$
%is $$E:= span\{e_1,\dots,e_k\}$$  %$$E := \{te_1 : t\in \mathbb{R}\}.$$ 
%\item Let $H$ be the orthogonal complement of $E$ so that $\mathbb{R}^n = E \oplus H.$ 
%\item Assume there is an $\epsilon > 0$ so the functions $g_r(te_i) = 0$ for all $t\in (-\epsilon, \epsilon)$, for all $r$ in $I$ and $i$ in ${1,\dots k}.$
%\item Assume that in $E$, $Df = 0$ and $Hf$ is negative definite.   %$\frac{\partial }{\partial t}f(0) = 0$, $\frac{\partial^2 }{\partial t^2}f(0) < 0.$
%\end{enumerate}
%\end{assumptions}

\begin{lemma}\label{lem1} Given Conditions \ref{assumpt}, the linear program 
$$\max_{x\in H }F(0)\cdot x\textrm{ subject to }G_r(0)\cdot x \ge 0, r\in I$$
has a unique maximum at $x = 0$
\end{lemma}
\begin{proof}
By conditions \ref{abounded} and \ref{asolutionset}, the linear program
$$\max_{x\in \mathbb{R}^n}F(0)\cdot x\textrm{ subject to }G_r(0)\cdot x \ge 0, r\in I$$
is maximized exactly on $E$. The feasible set $\{x : G_r(0)\cdot x \ge 0, r\in I \textrm{ and } x \in H\}$ is a subset of the feasible set $\{x:G_r(0)\cdot x \ge 0, r\in I\}$. Thus, the program 
$$\max_{x\in H}F(0)\cdot x\textrm{ subject to }G_r(0)\cdot x \ge 0, r\in I$$
 is maximized exactly on the non-empty intersection $$E\cap \{x :G_r(0)\cdot x\ge 0, r\in I\} \cap H = 0.$$
\end{proof}

\begin{definition}
A \emph{finitely generated cone} is a subset of $\mathbb{R}^n$ which is the non-negative span of a finite set of non-zero vectors $\{v_1,\dots, v_m\}$ in $\mathbb{R}^n$, which are called the \emph{generators} of the cone. 
\end{definition}
\begin{definition}
A \emph{conical linear program} is a linear program with a constraint set that is a finitely generated cone.
\end{definition}

The linear programs described throughout this section are always constrained to be on the intersection of half-spaces with $0$ on the boundary. These are conical programs.

\begin{definition}
For a cone $C$, the set $C^p := \{x \in \mathbb{R}^n: v\cdot x \le 0 \textrm{ for all } v \in C\}$ is the \emph{polar cone} of $C.$
\end{definition}

\begin{lemma}\label{lem2}
A conical linear program with $F\ne 0$ given by
$$\max_{x\in \mathbb{R}^n} F\cdot x\textrm{ subject to }G_r\cdot x \ge 0, r\in I $$ 
(a) has a unique\begin{footnote}{The maximum satisfies a stronger uniqueness condition. It is stable under perturbations of $F$ and $G_k$.}\end{footnote} maximum at $x= 0$ iff $F$ is in the interior of the polar cone $C^p$ of $C= \{x : G_r\cdot x \ge 0, r\in I\}$ (b) has a bounded solution iff $F$ is in the polar cone $C^p$ of  $C =\{x : G_r\cdot x \ge 0, r\in I\}$ and attains its maximum exactly on the span of the generators $v_i$ such that $F\cdot v_i =0.$
\end{lemma}
\begin{proof}
If $F$ is in the interior of the polar cone $C^p$, then $F\cdot v_i < 0$ for all generators $v_i$.  Therefore $F\cdot x$ is uniquely maximized in $C$ at the vertex. If $F$ is on the boundary of the polar cone, then $F\cdot x$ is maximized in $C$ exactly on the span of the generators $v_i$ for which $F\cdot v_i = 0$ as $F\cdot v_j < 0$ otherwise. If $F$ is outside the polar cone, then $F\cdot v_i >0$ for some generator $v_i$. Then $F\cdot x$ is unbounded in $C$.\end{proof}

\begin{lemma}\label{lem3} Given Conditions \ref{assumpt}, there exists $\epsilon > 0$ such that for all $t$ in  $(-\epsilon,\epsilon)$, the linear program 
$$\max_{y_t \in H}F(t)\cdot y_t$$
subject to 
$$G_r(t)\cdot y_t \ge 0, r \in I$$
has a unique maximum at $y_t = 0$.\begin{footnote}{Here $y_t$ is a dummy variable and does not depend on $t$. It is labeled $y_t$ to ease later exposition.}\end{footnote}
\end{lemma}
\begin{proof} The program for $t\in (-\epsilon,\epsilon)$, for $y_t$ in $H$, for each fixed $t$ in $(-\epsilon,\epsilon)$, for some $\epsilon> 0$, can be written as a conical program on all of $\mathbb{R}^n$ with a cone $C_t$ in $\mathbb{R}^n$ of co-dimension $\ge 1$ by introducing further constraints $e_1\cdot y_t \ge 0$ and $-e_1\cdot y_t \ge 0.$  By \ref{lem1} and \ref{lem2}, $F(0)$ is in the polar cone of $C_0 = \{y_0: G_r(0)\cdot y_0 \ge 0, e_1\cdot y_0 \ge 0, -e_1\cdot y_0 \ge 0\}.$   As $f,g_r \in C^\omega$, the condition of  $F(t)$ being in the interior of the polar cone $C_t^p$ is open and the condition of the feasible set $C_t = \{y_t: G_r(t)\cdot y_t \ge 0, e_1\cdot y_t \ge 0, -e_1\cdot y_t \ge 0\}$ being conical is open.\begin{footnote}{The relationships between the constraint cone, the generators $v_i$ and the constraint gradients $G_k$ is subtle, but the openness of the condition follows from the continuity of the distance function.}\end{footnote}  Therefore, by \ref{lem2} the program has a unique maximum at $y_t = 0$ for each fixed $t$ in $(-\epsilon,\epsilon)$ for some $\epsilon> 0$.\end{proof}

\begin{lemma}\label{lem4}
Given Conditions \ref{assumpt} and $\epsilon$ as in Lemma \ref{lem3}, for all $t \in (-\epsilon, \epsilon)$
there exists $\delta(t)>0$ and a cube $Q(t) \subset \mathbb{R}^n$ of side length $2\delta(t)$ such that 
$$\{(F(t) + Q(t)) \cap (\partial (C_t^p) + Q(t))\} = \emptyset.$$
\end{lemma}
\begin{proof}
This follows from \ref{lem3}, which shows $F(t)$ is in the interior of the polar cone $C_t^p$.  Then $F(t)$ and the boundary of $C_t^p$ can be separated and the existence of $Q$ is trivial.
\end{proof}

\begin{corollary}\label{cor1}
Given \ref{assumpt} and $\epsilon$ as in Lemma \ref{lem3}, for all $t \in (-\epsilon, \epsilon),$ 

$$(F(t)+\Delta) \cdot y_t \le 0$$
whenever $y_t$ satisfies
$$(G_r(t)+\Delta_r)\cdot y_t \ge 0, r\in I
\textrm{ and } 
e_1\cdot y_t \ge 0, -e_1\cdot y_t \ge 0$$
 where $\Delta$ and $\Delta_r$ are any points in the $2\delta(t)$-cube $Q(t)$ and $y_t$ is in $H$.
\end{corollary}
\begin{proof}
By \ref{lem4}, $F(t)+\Delta$ is in the interior of the polar cone $C_{t,\Delta}^p$, where $C_{t,\Delta} = \{y_t: (G_r(t)+\Delta_r)\cdot y_t \ge 0, e_1\cdot y_t \ge 0, -e_1\cdot y_t \ge 0, r\in I\}$.  
\end{proof}

\begin{lemma}\label{lem5}
Given Conditions \ref{assumpt} and $\epsilon$ as in Lemma \ref{lem3}, for all $t \in (-\epsilon, \epsilon)$, let $y_t = x-te_1 \in H$. Choose $\Delta = \Delta(y_t)$ and $\Delta_r=\Delta_r(y_t)$ in the $2\delta(t)$-cube $Q(t)$ to be the corner given by the sign of $x-te_1 = y_t.$ Then there is an $\epsilon_t$ for which 

$$(F(t)+\Delta(y_t))\cdot y_t \le 0 \implies f(x)-f(te_1) \le 0$$
and
$$(G_r(t)+\Delta_r(y_t))\cdot y_t \le 0 \implies g_r(x)-g_r(te_1) = g_r(x)\le 0$$

for all $\|y_t\| \le \epsilon_t$.
\end{lemma}
\begin{proof}
This follows from the local expansions of the nonlinear program.  By this choice of $\Delta(y_t)$ and $\Delta_r(y_t),$

$$f(x)-f(te_1)  = F(t)\cdot (x-te_1) + O(t^2) = F(t)\cdot y_t + O(t^2) $$
$$ \le  F(t)\cdot y_t + \delta(t) \|y_t\|_1 = (F(t)+\Delta(y_t))\cdot y_t $$
and using assumption \ref{anbhd},
$$g_r(x) = g_r(x)-g_r(te_1) =  G_r(t)\cdot (x-te_1) + O(t^2) = G_r(t)\cdot y_t +O(t^2) $$
$$ \le  G_r(t)\cdot y_t + \delta(t) \|y_t\|_1 = (G_r(t)+\Delta_r(y_t))\cdot y_t.$$
\end{proof}

By Lemma \ref{lem4} and Corollary \ref{cor1}, for $t$ in $(-\epsilon,\epsilon)$, the program  $$\max_{y_t \in H} (F(t) + \Delta)\cdot y_t\textrm{ subject to } (G_r + \Delta_r)\cdot y_t$$ is uniquely maximized at $y_t=0$ for any choice of $\Delta$, $\Delta_r$ in the $2\delta(t)$ cube $Q(t)$. Combined with \ref{lem5}, there is an $\epsilon_t$ neighborhood of $0$ where $f(y_t+te_1)$ is less than $f(te_1)$ on $\cup_{\Delta_r \in Q(t)}\{y_t: (G_r+\Delta_r) \cdot y_t \ge 0, r \in I, y_t \in H\}$, which contains the feasible set  $\{y_t: g_r(y_t+te_1) \ge 0, r\in I, y_t \in H\}$. Therefore the nonlinear programs $f(y_t+te_1)$ subject to $g_r(y_t+te_1)\ge 0$, $y_t \in H$, which are parameterized by $t$ in $(-\epsilon, \epsilon)$, have local maxima at $y_t=0$.  This gives the following:

\begin{theorem}\label{th1}
Given Conditions \ref{assumpt}, a fixed $t$ in $(-\epsilon, \epsilon)$ and choosing $\Delta$ and $\Delta_r$ as in Lemma \ref{lem5}, for $x$ satisfying $g_r(x)\ge 0$ for all $r$ in $I$ and $y_t = x-te_1$ in $H$, 
there exist linear programs\begin{footnote}{These programs may depend on a choice of $y_t \in H$, but $f(x)$ is always less then $f(te_1)$ by Lemma \ref{lem5}.}\end{footnote} $$\max_{y_t \in H}(F(t)+\Delta(y_t))\cdot y_t \textrm{ subject to }  (G_r(t)+\Delta_r(y_t))\cdot y_t \ge 0$$
that give solutions to the nonlinear programs $$\max_{x \in H+te_1} f(x)  \textrm{ subject to }   g_r(x)\ge 0$$
in an $\epsilon_t$ neighborhood of $te_1$ in $H+te_1.$ \qed
\end{theorem}

By choice of a sufficiently small $\epsilon$ and a minimal\begin{footnote}{This exists by a compactness argument.}\end{footnote} non-zero $\epsilon_t$, Theorem \ref{th1} gives an open neighborhood of $0$ in which the maximum value of the original nonlinear program occurs on $E$. The conditions for the first and second $t$-derivatives at $0$ shows $0$ to be a local maximum for the nonlinear program
 $$\max_{x\in\mathbb{R}^n}f(x) \textrm{ subject to }  g_r(x) \ge 0.$$

\begin{theorem}
A nonlinear program satisfying Conditions \ref{assumpt} has an isolated local maximum at 0 with f(0) = 0. \qed
\end{theorem}






\bibliography{ngon}
\end{document}  

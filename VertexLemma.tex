\documentclass[11pt, oneside]{article}   	% use "amsart" instead of "article" for AMSLaTeX format
\usepackage{geometry}                		% See geometry.pdf to learn the layout options. There are lots.
\geometry{letterpaper}                   		% ... or a4paper or a5paper or ... 
%\geometry{landscape}                		% Activate for for rotated page geometry
%\usepackage[parfill]{parskip}    		% Activate to begin paragraphs with an empty line rather than an indent
\usepackage{graphicx}				% Use pdf, png, jpg, or eps� with pdflatex; use eps in DVI mode
								% TeX will automatically convert eps --> pdf in pdflatex		
\usepackage{amssymb}
\usepackage{amsmath}
\usepackage{amsthm}
\usepackage [autostyle, english = american]{csquotes}

\newtheorem{definition}{Definition}[section]
\newtheorem{theorem}{Theorem}[section]
\newtheorem{conjecture}{Conjecture}[section]
\newtheorem{lemma}{Lemma}[section]
\newtheorem{conj}{Conjecture}[section]
\newtheorem{corollary}{Corollary}[section]
\newtheorem{remark}{Remark}[section]
\newtheorem{proposition}{Proposition}[section]
\newtheorem{conditions}{Conditions}[section]

%\newcommand{\defv}[1]{\textbf{\textit{#1}}}


\title{The local optimality of the double lattice packing}
\author{Yoav Kallus and W\"oden Kusner}
%\date{}							% Activate to display a given date or no date

\begin{document}
\bibliographystyle{plain}
%\maketitle
%\abstract
%This paper shows that the dense double lattice construction of Kuperberg and Kuperberg is locally optimal for polygons in the full space of packings.




\section{The Vertex-Vertex Lemma}

By the construction of 


there do exist examples of convex bodies that have affine diameters ...


\begin{proposition}
The affine diameter of a convex polygon meets a vertex or is equivalent to one.
\end{proposition}
\begin{proof}
parallel transport or transport in the increasing direction of the edge cone.
\end{proof}


\begin{proposition}
Quadrilateral configuration


The affine diameter meets only one vertex, unless 

\end{proposition}

\begin{proof}

Consider the case where an affine diameter meets two vertices.  

If the affine diameter is not uniquely determined then the vertex-vertex condition is also an edge-edge configuration.

Otherwise, one of the vertices may be fixed and swept out as an affine diameter. 

\end{proof}

\begin{proposition}
The vertices of the half length parallelogram meet no vertices unless the stability condition is degenerate.



\end{proposition}

\begin{proposition}

\end{proposition}


affine diameter

perturbation of quadrangle vertices

depends on analysis of the the stability criterion of slopes

sketch at a single vertex...

intermediate value theorem




\bibliography{ngon}
\end{document}  